\section{Podsumowanie}

Dołączenie CMA w~czerwcu 2012 r.\ do Linuksa w~żadnym stopniu nie
oznacza zakończenia nad nim prac.  Nadal jest wiele właściwości, które
można usprawnić i~nowych funkcji, które można dodać.

Najistotniejszym aspektem CMA jest czas potrzebny na zrealizowanie
alokacji.  Migracja tysięcy stron może być czasochłonna, gdyż nawet
w~systemach z~szybką magistralą, kopiowanie kilkudziesięciu megabajtów
danych może trochę potrwać.

\textcite{patch:cma-discard} zaimplementował we wrześniu 2012
r.\ usprawnienie, które w~istotny sposób skraca czas alokacji.
W~swoich testach zauważył przyśpieszenie alokacji \unit[10]{MiB}
z~\unit[146]{ms} do zaledwie \unit[7]{ms}.  Pomysł polega na
odrzucaniu stron które można w~prosty sposób odzyskać.  Najprostszym
przykładem są tutaj bufory dyskowe -- ich zawartość można przywrócić
ponownie odczytując dane z~nośnika.

Innym możliwym usprawnieniem CMA jest ulepszenie algorytmu doboru
zakresu stron.  Obecnie stosowana metoda „pierwszy pasujący” nie
uwzględnienia, które strony wymagają migracji.  Może to powodować, iż
alokator będzie dokonywał migracji, którym można było zapobiec.  Na
chwilę obecną nie wiadomo jednak, czy zysk z~nowego algorytmu nie
zostałby przysłonięty kosztami wynikającymi z~jego złożoności jak
i~możliwą większą fragmentacją.

Innym dość uciążliwym problemem, z~którym CMA musi sobie radzić jest
fakt, że strony ruchome nie zawsze można migrować, co może być
spowodowane brakiem funkcji migrującej lub chwilowym wykorzystaniem
strony w~kontekście wymagającym stałego adresu fizycznego.

Do pierwszej kategorii należą np.\ strony wykorzystywane przez wiele
systemów plików.  Nawet w~bardzo popularnym i~powszechnie używanym
systemie plików ext4, strony dziennika nie posiadają zaimplementowanej
funkcji migracji.

Druga kategoria to sytuacje, gdy strona została unieruchomiona na
czas, gdy wykonywany jest na niej transfer DMA.  Przykładowo, jeżeli
dane są kopiowane pomiędzy taką stroną, a~dyskiem twardym.  Jednym
z~rozważanych przeze mnie rozwiązań było migrowanie strony poza region
CMA zanim zostanie ona unieruchomiona, ale niestety sytuacja taka jest
na tyle powszechna, że degradacja wydajności byłaby zbyt duża.

Innymi możliwymi rozwiązaniami jest ograniczenie wykorzystywania
regionów CMA tylko do stron przeznaczonych dla takich funkcji jak zRam
\autocite{patch:zram}, pamięć transcendentna
\autocite{bib:transcendent}, czy mechanizm \code|POSIX_FADV_VOLATILE|
\autocite{bib:fadv-volatile}.  Gdyby to zrobić, istniałaby większa
gwarancja, iż dane można przenieść, a~w~przypadku dwóch ostatnich
mechanizmów, dane można by zwyczajnie odrzucić.  Z~drugiej strony
w~znacznym stopniu zmniejszyłoby to użyteczność stron z~regionu CMA,
potencjalnie do tego stopnia, iż przez większość czasu nie byłyby one
zupełnie wykorzystywane.

Widać, że droga przed alokatorem CMA jest otwarta i~istnieje wiele
aspektów, które można ulepszać, a~dzięki coraz większemu gronu osób
zainteresowanych tym kodem, można pokusić się o~predykcję, iż
mechanizm CMA będzie się rozwijał.  Równocześnie kolejne osoby
wykazują zainteresowanie korzystania z~niego w~systemach znacznie
różniących się od platform, dla których był projektowany
(tj.\ telefonów komórkowych), takich jak zarządcy maszyn wirtualnych
jak i~oprogramowanie samolotów.
