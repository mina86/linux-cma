\begin{titlepage}
    % Strona tytułowa
    \vbox to\textheight{\hyphenpenalty=10000
    \begin{center}
        \begin{tabular}{p{107mm} p{9cm}}
            \begin{minipage}{9cm}
              \begin{center}
              Politechnika Warszawska \\
              Wydział Elektroniki i~Technik Informacyjnych \\
              Instytut Informatyki
              \end{center}
            \end{minipage}
            &
            \begin{minipage}{8cm}
            \begin{flushleft}
             \footnotesize
              Rok akademicki 2012/2013
            \vspace*{2.75\baselineskip}
            \end{flushleft}
            \end{minipage} \\
        \end{tabular}
        \vspace*{3.75\baselineskip}
        \par\vspace{\smallskipamount}
        \includegraphics[width=4cm]{build/pw.eps}\par
        \vspace*{2\baselineskip}{\LARGE Praca dyplomowa inżynierska\par}
        \vspace{3\baselineskip}{\LARGE\strut \theauthor\par}
        \vspace*{2\baselineskip}{\huge\bfseries \thetitle\par}

        \vspace*{7\baselineskip}
        \hfill\mbox{}\par\vspace*{\baselineskip}\noindent
        \begin{tabular}[b]{@{}p{3cm}@{\ }l@{}}
            {\large\hfill } & {\large }
        \end{tabular}
        \hfill
        \begin{tabular}[b]{@{}l@{}}
        Opiekun pracy: \\[\smallskipamount]
        {\large dr inż.\ Wojciech Zabołotny}
        \end{tabular}\par
        \vspace*{4\baselineskip}
    \begin{tabular}{p{\textwidth}}
    \begin{flushleft}
        \begin{minipage}{7cm}
        Ocena \dotfill
        \par\vspace{1.6\baselineskip}
        \dotfill
        \par\noindent
        \centerline{\footnotesize Podpis Przewodniczącego} \par
        \centerline{\footnotesize Komisji Egzaminu Dyplomowego}\par
        \end{minipage}
    \end{flushleft}
    \end{tabular}
    \end{center}}

    % Życiorys
    \newpage\thispagestyle{empty}
    \begin{tabular}{p{5cm} p{12cm}}
    \begin{minipage}{5cm}
    \center
    \IfFileExists{bsc-photo.eps}{
      \includegraphics[width=4.5cm]{bsc-photo.eps}
    }{
      \includegraphics[width=4.5cm]{build/photo.eps}
    }
    \end{minipage}
    &
    \begin{minipage}{12cm}
    \begin{flushleft}
    \par\noindent\vspace{1\baselineskip}
    \begin{tabular}[h]{l l}
    {\it Kierunek:}       & Informatyka \\[12pt]
    {\it Specjalność:}    & Inżynieria Systemów \\
                          & Informatycznych \\[12pt]
    {\it Data urodzenia:} & 31 sierpnia 1986~r. \\[12pt]
    {\it Data rozpoczęcia studiów:} & 1 lutego 2006 r. \\
    \end{tabular}
    \par\noindent\vspace{1\baselineskip}
    \end{flushleft}
    \end{minipage}
    \end{tabular}
    \vspace*{1\baselineskip}
    \begin{center}
        {\large\bfseries Życiorys}\par\bigskip
    \end{center}

    \IfFileExists{bsc-bio.tex}{\indent\input{bsc-bio}}{
      \begin{center}
        $\ddot\smile$
      \end{center}
    }
    \par
    \vspace{2\baselineskip}
    \hfill\parbox{15em}{{\small\dotfill}\\[-.3ex]
    \centerline{\footnotesize podpis studenta}}\par
    \vspace{3\baselineskip}
    \begin{center}
        {\large\bfseries Egzamin dyplomowy} \par\bigskip\bigskip
    \end{center}
    \par\noindent\vspace{\baselineskip}
    Złożył egzamin dyplomowy w dn. \dotfill
    \par\noindent\vspace{\baselineskip}
    Z wynikiem \dotfill
    \par\noindent\vspace{\baselineskip}
    Ogólny wynik studiów \dotfill
    \par\noindent\vspace{\baselineskip}
    Dodatkowe wnioski i uwagi Komisji \dotfill
    \par\noindent\vspace{\baselineskip}
    \dotfill

    % Streszczenie
    \newpage\thispagestyle{empty}
    \vspace*{2\baselineskip}
    \begin{center}
        {\large\bfseries Streszczenie}\par\bigskip
    \end{center}

    {\itshape Wiele podzespołów komputera, a~szczególnie
      tzw.\ systemów wbudowanych, jest zazwyczaj podłączonych
      bezpośrednio do magistrali systemowej, przez co muszą operować
      adresami fizycznymi.  Równocześnie mechanizmy bezpośredniego
      dostępu do pamięci (\ang{Direct Memory Access}, \acc{DMA}) są
      ograniczone do transferów sekwencyjnych.  Stwarza to potrzebę,
      alokowania dużych ciągłych fizycznie obszarów pamięci do
      wykorzystania w~takich układach.

      Dotychczas stosowane w~systemach bazujących na jądrze Linux
      rozwiązania wiążą się z~rezerwowaniem dużego obszaru pamięci,
      który wyjęty spod kontroli Linuksa przestaje być użyteczny dla
      jądra i~w~rezultacie jest wykorzystywany nieefektywnie.

      Praca opisuje stworzony przeze mnie alokator pamięci ciągłej,
      \ang{Contiguous Memory Allocator} (\acc{CMA}), który rozwiązuje ten
      problem poprzez zastosowanie mechanizmu migracji, który pozwala
      przenosić zajęte strony i~w~ten sposób tworzyć długie sekwencje
      wolnych stron.}

    \vspace*{1\baselineskip}

    \noindent{\bf Słowa kluczowe}: {\itshape alokacja pamięci, Linux,
      systemy wbudowane.}
    \par
    \vspace{4\baselineskip}
    \begin{center}
        {\large\bfseries Abstract}\par\bigskip
    \end{center}
    \noindent{\bf Title}: {\itshape \theengishtitle}\par
    \vspace*{1\baselineskip}

    {\itshape Many computer components, especially in a~so called
      embedded system, are attached directly to the system bus and thus
      need to operate on physical addresses.  At the same time, Direct
      Memory Access (\acc{DMA}) is limited to sequential transfers only.
      This creates a need to allocate big physically contiguous memory
      buffers to be used with such components.

      Solutions previously used in Linux-based systems boil down to
      reserving a~big memory area which exempted from Linux control
      cannot be used efficiently by the kernel.

      This work describes Contiguous Memory Allocator
      (\acc{CMA})\,---\,an allocator written by me which solves the
      problem by using migration, which makes it possible to move
      allocated pages and thus create a~long sequence of free pages.}
    \vspace*{1\baselineskip}

    \noindent{\bf Key words}: {\itshape memory allocation, Linux,
      embedded systems.}

\end{titlepage}
