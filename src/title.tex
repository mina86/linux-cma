\begin{titlepage}
    % Strona tytułowa
    \vbox to\textheight{\hyphenpenalty=10000
    \begin{center}
	\begin{tabular}{p{107mm} p{9cm}}
	    \begin{minipage}{9cm}
	      \begin{center}
	      Politechnika Warszawska \\
	      Wydział Elektroniki i~Technik Informacyjnych \\
	      Instytut Informatyki
	      \end{center}
	    \end{minipage}
	    &
	    \begin{minipage}{8cm}
	    \begin{flushleft}
	     \footnotesize
	      Rok akademicki 2011/2012
	    \vspace*{2.75\baselineskip}
	    \end{flushleft}
	    \end{minipage} \\
	\end{tabular}
	\vspace*{3.75\baselineskip}
	\par\vspace{\smallskipamount}
	\vspace*{2\baselineskip}{\LARGE Praca dyplomowa inżynierksa\par}
	\vspace{3\baselineskip}{\LARGE\strut \theauthor\par}
	\vspace*{2\baselineskip}{\huge\bfseries \thetitle\par}

	\vspace*{7\baselineskip}
	\hfill\mbox{}\par\vspace*{\baselineskip}\noindent
	\begin{tabular}[b]{@{}p{3cm}@{\ }l@{}}
	    {\large\hfill } & {\large }
	\end{tabular}
	\hfill
	\begin{tabular}[b]{@{}l@{}}
	Opiekun pracy: \\[\smallskipamount]
	{\large dr inż.\ Wojciech Zabołotny}
	\end{tabular}\par
	\vspace*{4\baselineskip}
    \begin{tabular}{p{\textwidth}}
    \begin{flushleft}
	\begin{minipage}{7cm}
	Ocena \dotfill
	\par\vspace{1.6\baselineskip}
	\dotfill
	\par\noindent
	\centerline{\footnotesize Podpis Przewodniczącego} \par
	\centerline{\footnotesize Komisji Egzaminu Dyplomowego}\par
	\end{minipage}
    \end{flushleft}
    \end{tabular}
    \end{center}}

    % Życiorys
    \newpage\thispagestyle{empty}
    \begin{tabular}{p{5cm} p{12cm}}
    \begin{minipage}{5cm}
    \center
    \end{minipage}
    &
    \begin{minipage}{12cm}
    \begin{flushleft}
    \par\noindent\vspace{1\baselineskip}
    \begin{tabular}[h]{l l}
    {\normalsize\it Specjalność:} & Informatyka -- \\
    & Inżynieria oprogramowania \\
    & i~systemy informacyjne
    \end{tabular}
    \par\noindent\vspace{1\baselineskip}
    \begin{tabular}[h]{l l}
    {\normalsize\it Data urodzenia:} & {\normalsize 31 sierpnia 1986~r.}
    \end{tabular}
    \par\noindent\vspace{1\baselineskip}
    \begin{tabular}[h]{l l}
    {\normalsize\it Data rozpoczęcia studiów:} & {\normalsize 1 lutego 2006 r.}
    \end{tabular}
    \par\noindent\vspace{1\baselineskip}
    \end{flushleft}
    \end{minipage}
    \end{tabular}
    \vspace*{1\baselineskip}
    \begin{center}
	{\large\bfseries Życiorys}\par\bigskip
    \end{center}

    \indent
    Nazywam się \theauthor \dots \TODO{do wypełnienia}
    \par
    \vspace{2\baselineskip}
    \hfill\parbox{15em}{{\small\dotfill}\\[-.3ex]
    \centerline{\footnotesize podpis studenta}}\par
    \vspace{3\baselineskip}
    \begin{center}
 	{\large\bfseries Egzamin dyplomowy} \par\bigskip\bigskip
    \end{center}
    \par\noindent\vspace{1.5\baselineskip}
    Złożył egzamin dyplomowy w dn. \dotfill
    \par\noindent\vspace{1.5\baselineskip}
    Z wynikiem \dotfill
    \par\noindent\vspace{1.5\baselineskip}
    Ogólny wynik studiów \dotfill
    \par\noindent\vspace{1.5\baselineskip}
    Dodatkowe wnioski i uwagi Komisji \dotfill
    \par\noindent\vspace{1.5\baselineskip}
    \dotfill

    % Streszczenie
    \newpage\thispagestyle{empty}
    \vspace*{2\baselineskip}
    \begin{center}
	{\large\bfseries Streszczenie}\par\bigskip
    \end{center}

    {\itshape Praca ta prezentuje sposób alokacji dużych obszarów
      ciągłej fizycznie pamięci w~systemach opartych na jądrze Linux.
      Zastosowanie w~procesorach jednostek zarządzania pamięcią
      pozwala na uniknięcie konieczności takich alokacji, jednak
      w~systemach wbudowanych poszczególne komponenty są często
      pozbawiony jednostki transakcji adresów pamięci, przez co muszą
      operować bezpośrednio na adresach fizycznych.  Aby temu
      zaradzić, stworzyłem alokator pamięci ciągłej (ang. {\it
        Contiguous Memory Allocator}, CMA), który pozwala przydzielać
      duże ciągłe fizycznie obszary pamięci, bez konieczności
      rezerwowania na wyłączność dużych obszarów przy starcie systemu.}
    \vspace*{1\baselineskip}

    \noindent{\bf Słowa kluczowe}: {\itshape alokacja pamięci, Linux,
      systemy wbudowane.}
    \par
    \vspace{4\baselineskip}
    \begin{center}
	{\large\bfseries Abstract}\par\bigskip
    \end{center}
    \noindent{\bf Title}: {\itshape \theengishtitle}\par
    \vspace*{1\baselineskip}

    {\itshape This thesis describes a~way to allocate large physically
      contiguous memory regions in Linux-based systems.  Memory
      management units allows to avoid the need of such allocations,
      but in embedded systems, individual components often lack an
      address translation unit and thus need to operate directly on
      physical addresses.  To address this issue, I~have created
      a~Contiguous Memory Allocator (or CMA) which allows to allocate
      big physically contiguous memory regions without the need to
      reserve memory for exclusive use during system start.}
    \vspace*{1\baselineskip}

    \noindent{\bf Key words}: {\itshape memory allocation, Linux,
      embedded systems.}

\end{titlepage}
