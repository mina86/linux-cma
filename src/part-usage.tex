% Copyright (c) 2012 by Michał Nazarewicz <mina86@mina86.com>
% Distributed under the terms of the Creative Commons
% Attribution-ShareAlike 3.0 Unported (CC BY-SA 3.0) license.

\subsection{Using CMA from device drivers}

\begin{frame}
  \frametitle{Using CMA from device drivers}

  \begin{itemize}
  \item CMA is integrated with the DMA API.
  \item If device driver uses the DMA API, nothing needs to be changed.
  \item In fact, device driver should always use the DMA API and never
    call CMA directly.
  \end{itemize}
\end{frame}

\begin{frame}[fragile]
  \frametitle{Allocating memory from device driver}

  \begin{block}{Allocation}
\begin{lstlisting}
void *my_dev_alloc_buffer(
    unsigned long size_in_bytes, dma_addr_t *dma_addrp)
{
    void *virt_addr = dma_alloc_coherent(
        my_dev, size_in_bytes, dma_addrp, GFP_KERNEL);
    if (!virt_addr)
        dev_err(my_dev, "Allocation failed.");
    return virt_addr;
}
\end{lstlisting}
  \end{block}

\end{frame}

\begin{frame}[fragile]
  \frametitle{Releasing memory from device driver}

  \begin{block}{Freeing}
\begin{lstlisting}
void *my_dev_free_buffer(
    unsigned long size, void *virt, dma_addr_t dma)
{
    dma_free_coherent(my_dev, size, virt, dma);
}
\end{lstlisting}
  \end{block}
\end{frame}

\begin{frame}
  \frametitle{Documentation}

  \begin{itemize}
  \item \code{Documentation/DMA-API-HOWTO.txt}
  \item \code{Documentation/DMA-API.txt}
  \item Linux Device Drivers, \nth{3} edition, chapter 15.
    \begin{itemize}
    \item \url{http://lwn.net/Kernel/LDD3/}
    \end{itemize}
  \end{itemize}
\end{frame}
