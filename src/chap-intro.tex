\section{Opis problemu}

W~celu zwiększenia efektywności działania oraz liczby udostępnianych
funkcji, komputery a w~szczególności telefony komórkowe posiadają
wiele wyspecjalizowanych podzespołów.  W~wielu przypadkach, procesor
komunikuje się z~nimi poprzez bufory w pamięci operacyjnej,
przekazując do urządzenia jedynie adresy gdzie dane się znajdują.
Dostęp do \acc{RAM}-u poprzez takie podzespoły może się jednak wiązać
z~wieloma ograniczeniami.

\begin{figure}[tbp]
  \centering
  \subfloat[System z~układem \acc{MMU} pomiędzy procesorem a~pamięcią oraz
    urządzeniami podłączonymi bezpośrednie do szyny pamięci.]{
    \label{fig:sys-with-mmu}
    \includegraphics[width=.25\textwidth]{build/mmu-iommu-images--img-nommu.eps}
  } \qquad
  \subfloat[Systemu z~kontrolerem \acc{DMA}, który pośredniczy w~transferach
    danych pomiędzy pamięcią i~urządzeniami.]{
    \label{fig:sys-with-dma}
    \includegraphics[width=.25\textwidth]{build/mmu-iommu-images--img-dma.eps}
  }\qquad
  \subfloat[Systemu z~zarówno układem \acc{MMU} jak i~\acc{IOMMU}, które tłumaczą
    adresy widziane przez odpowiednio procesor oraz urządzenia.]{
    \label{fig:sys-with-iommu}
    \includegraphics[width=.25\textwidth]{build/mmu-iommu-images--img-iommu.eps}
  }
  \caption[Różne przestrzenie adresowe dostępne
    w~komputerze.]{Reprezentacja systemów z~różnymi podzespołami
    uczestniczącymi w~translacji adresów lub transferach danych do
    pamięci operacyjnej.}
  \label{fig:mmu-iommu}
\end{figure}

Nowoczesne architektury przeznaczone do serwerów i~komputerów
osobistych posiadają jednostkę zarządzania pamięcią (\ang{memory
  management unit}, \acc{MMU}), która tłumaczy adresy logiczne na fizyczne.
Dzięki temu, bufory ciągłe z~punktu widzenia procesora, mogą
w~rzeczywistości być podzielone na wiele części rozrzuconych po
pamięci fizycznej.  W~ten sposób, nawet jeżeli program alokuje
wielomegabajtowy obszar, system może zrealizować żądanie alokując
czterokibibajtowe strony i~nie przejmować się fragmentacją pamięci.

Niemniej, jak to przedstawia rysunek \subref*{fig:sys-with-mmu},
jednostka \acc{MMU} nie jest zazwyczaj dostępny dla pozostałych układów
znajdujących się w~urządzeniu, takich jak karta dźwiękowa, czy
kontroler sieciowy.  Także i~dla tych przypadków istnieje rozwiązanie
w~postaci mechanizmu bezpośredniego dostępu do pamięci (\ang{Direct
  Memory Access}, \acc{DMA}), którego celem jest odciążenie procesora od
przesyłania danych.  Co prawda urządzenie nadal znajduje się
w~przestrzeni adresów fizycznych, co ilustruje rysunek
\subref*{fig:sys-with-dma}, ale dzięki układowi \acc{DMA} nieciągłość
buforów może zostać przed nim ukryta.

Niestety mechanizm bezpośredniego dostępu do pomięci jest ograniczony
do transferów sekwencyjnych.  Sprawdza się bardzo dobrze dla operacji
dyskowych, ale nie nadaje się dla sprzętowego dekodera wideo, który
potrzebuje dostępu do wielu dekodowanych ramek
jednocześnie.\footnote{Jednym z~rodzajów ramek stosowanych do
  kodowania klatki ze strumienia wideo jest b-ramka, która już nawet
  w~starszych standardach takich jak MPEG-2 może odwoływać się do
  jednej poprzedzającej i~jednej następującej klatki, a w~przypadku
  nowszego standardu H.264, może zależeć od więcej niż dwóch innych
  ramek.}

Oczywiście nie ma żadnych technologicznych przeszkód do zastosowania
jednostki translacji adresów również dla podzespołów innych niż
procesor, tak jak to przedstawia rysunek \subref*{fig:sys-with-iommu}.
Istotnie istnieją platformy sprzętowe z~\acc{MMU} wejścia/wyjścia
(ang.\ \acc{IOMMU}), który pozwala na budowanie dużych buforów złożonych ze
stosunkowo małych stron.  W~takich systemach, w~zasadzie nie ma (lub
nie powinno być) konieczności alokowania wielomegabajtowych ciągłych
obszarów pamięci.

Niestety, nawet jeżeli \acc{IOMMU} jest dostępne na danej platformie, jego
obecność może się wiązać z~dodatkowym kosztem wynikającym
z~nieoptymalnego kodu \autocite{bib:price-of-safety} lub konieczności
odczytywania mapowań z~pamięci
\autocite{bib:mitigate-iotlb-bottleneck}.  Z~tego względu architekt
platformy może zdecydować, aby wyłączyć \acc{IOMMU} i~rozwiązać problem
„w~oprogramowaniu”.

Z~uwagi na koszty i~ograniczenia zarówno kontrolerów \acc{DMA} jak
i~układów \acc{MMU}, w~wielu systemach wbudowanych, takich jak
np.\ telefony komórkowe, takie mechanizmy są często niedostępne.
Jednocześnie, właśnie takie urządzenia posiadają dużo
wyspecjalizowanych podzespołów, jak chociażby aparat fotograficzny,
czy układ do dekodowania obrazów \acc{JPEG}.

Powoduje to, że tego typu układy muszą operować bezpośrednio na
adresach fizycznych i~w~konsekwencji, wszelkie stosowane przez nie
bufory muszą być ciągłe.  Niestety Linux nie jest dobrze przystosowany
do alokowania takich obszarów.\footnote{W~szczególności, Linux nie
  jest nawet w~stanie (bez modyfikowania źródła) zarządzać obszarami
  większymi niż cztery mebibajty (1024 strony), gdy tymczasem
  pięciomegapikselowa kamera potrzebuje buforu o~rozmiarze 15
  megabajtów, a~pojedyncza ramka \textit{full \acc{HD}} (tj.\ 1920 $\times$
  1080) zajmuje ponad sześć megabajtów.}


\section{Możliwe rozwiązania}

Ponieważ problem jest znany od dawna, na przestrzeni lat powstało
wiele rozwiązań programowych umożliwiających obejście trudność
w~alokacji dużych obszarów.  W~tym podrozdziale opiszę je pokrótce
oraz przedstawie ich ograniczenia.

\subsection{Przypisywanie pamięci na stałe}

Najprostszym, i~stosunkowo często stosowanym, rozwiązaniem jest
rezerwacja przy starcie systemu pewnego regionu pamięci na potrzeby
konkretnych sterowników.

Najłatwiejszym, acz niezbyt eleganckim sposobem jest wykorzystanie
argumentu \code|mem| jądra.  Przekazany przez program startujący
powoduje, że Linux nie stara się automatycznie wykryć dostępnej
pamięci \acc{RAM} i~zamiast tego interpretuje przekazane informacje.  W~ten
sposób, możliwe jest ograniczenie widzianej przez system pamięci, tak
że ukryte regiony mogą być wykorzystywane przez konkretne sterowniki.

Bardziej eleganckim rozwiązaniem jest skorzystanie z~alokatora
memblock, który jest aktywny zanim jądro zainicjuje wszystkie swoje
podsystemy.  Jego zadaniem jest śledzenie wolnej pamięci zanim
bardziej zaawansowany alokator stron będzie dostępny w~systemie.
Wołany dostatecznie wcześnie, jest w~stanie zaalokowanie duże obszary
pamięci, które potem można wykorzystać w~dowolny sposób.

Niestety, o~ile tego typu rozwiązania mogą być wystarczające, jeżeli
podzespoły wymagają stosunkowo małych buforów, przestają się skalować
przy współczesnych systemach, gdyż wymagają rezerwacji wielu
megabajtów, które przez większość czasu nie są do niczego
wykorzystywane.

\subsection{Pula pamięci fizycznej}

Bardziej skomplikowanym rozwiązaniem jest mechanizm, który rezerwuje
pewną przestrzeń pamięci, ale zamiast na stałe przypisywać obszary do
urządzeń, pozwala sterownikom alokować bufory wtedy, gdy są one
potrzebne.

W~trakcie moich prac stworzyłem \textit{Physical Memory Manager} (\acc{PMM})
\autocite{patch:pmm}, który implementuje dokładnie te założenia. W~tym
podstawowym zastosowaniu, \acc{PMM} nie przedstawia sobą nic nowego.  Już
bowiem w~1996 roku Matt Welsh napisał pierwszą wersję patcha
\emph{bigphysarea} dla jądra 1.3.71, który był z~różnym zaangażowaniem
utrzymywany i~przystosowywany aż do wersji 3.2 Linuksa
\autocite{patch:bigphys}.

Mechanizm \acc{PMM} umożliwiał jednak alokowanie dużych obszarów
ciągłej pamięci fizycznej nie tylko sterownikom działającym
w~przestrzeni jądra, ale także programom działającym pod kontrolą
systemu. Co więcej, dzięki integracji z~mechanizmem współdzielenia
pamięci Systemu V~(tj.\ funkcjami \code|shmget|, \code|shmat|,
\code|shmdt| itp.)  wykorzystywanym między innymi przez X~Window,
\acc{PMM} pozwalał na dekodowanie obrazów i~strumieni wideo
bezpośrednio do buforów dostępnych dla serwera X11.  Dzięki temu,
w~całym procesie dane nie były niepotrzebnie kopiowane, co
minimalizowało użycie procesora i~szyny pamięci przyśpieszając
działanie systemu.

Jednakże, pamięć zarezerwowana przez \acc{PMM} i~tak przez większość czasu
była zupełnie nieużywana, a~więc marnowana.  Z~tego powodu, \acc{PMM} nie
został przyjęty przez społeczność programistów Linuksa i~musiałem
rozwijać inne rozwiązanie.

\subsection{Zarys Contiguous Memory Allocatora}

W~ten sposób zrodził się alokator pamięci ciągłej, \textit{Contiguous
  Memory Allocator} (\acc{CMA}), który umożliwia systemowi używanie
zarezerwowanej pamięci, o~ile żadne urządzenie jej w~danym momencie
nie potrzebuje.

Pierwsze wersje alokatora \acc{CMA} skupiały się w~dużej mierze na
umożliwianiu sterownikom alokowania różnych buforów w~różnych
obszarach pamięci.  Było to potrzebne, gdyż dekoder wideo stosowany na
platformie S5PV110 wymagał, aby różne dane były przechowywane
w~różnych bankach pamięci.  Pozwalało to na zwiększenie szybkości
dostępu do tych danych dzięki zastosowaniu odczytu z~dwóch banków
pamięci jednocześnie.

Z~czasem, coraz bardziej integrowałem mechanizm \acc{CMA} z~kodem
zarządzania pamięci w~Linuksie w~wyniku czego, pamięć rezerwowana przy
starcie, stała się dostępna dla reszty systemu, o~ile żaden sterownik
jej nie używał \autocite{patch:cma-24}.  Takie rozwiązanie zostało
ostatecznie zaakceptowane przez społeczność deweloperów Linuksa i~jest
dostępne począwszy od wersji 3.5 jądra.
