\chapter{Wstęp}

Tematem niniejszej pracy inżynierskiej jest sterownik dla jądra Linux,
który pozwala w~efektywny sposób alokować duże obszary ciągłej
fizycznie pamięci.  Opisanym mechanizmem jest stworzony przeze mnie
menadżer ciągłej pamięci, {\it Contiguous Memory Allocator} (lub CMA).

Podstawowym zastosowaniem dużych buforów, który brałem w~głównej
mierze podczas pisania alokatora CMA, jest wykorzystanie ich
w~podzespołach dostępnych w~nowoczesnych telefonach komórkowych.
Niemniej odkąd stworzony przeze mnie kod został dołączony do Linuksa,
różne osoby wykazały zainteresowanie, aby wykorzystywać go również
w~innych celach.


\section{Opis problemu}

W~celu zwiększenia efektywności działania oraz liczby udostępnianych
funkcji, komputery a w~szczególności telefony komórkowe posiadają
wiele wyspecjalizowanych podzespołów.  W~wielu przypadkach, procesor
komunikuje się z~nimi poprzez bufory w pamięci operacyjnej,
przekazując do urządzenia jedynie adresy gdzie dane się znajdują lub
powinny być zapisane.  Dostęp do RAM-u poprzez takie podzespoły może
się jednak wiązać z~wieloma ograniczeniami.

\begin{figure}[tbp]
  \centering
  \subfloat[System z~układem MMU pomiędzy procesorem a~pamięcią oraz
    urządzeniami podłączonymi bezpośrednie do szyny pamięci.]{
    \label{fig:sys-with-mmu}
    \includegraphics[width=.25\textwidth]{build/mmu-iommu-images--img-nommu.eps}
  } \qquad
  \subfloat[Systemu z~kontrolerem DMA, który pośredniczy w~transferach
    danych pomiędzy pamięcią i~urządzeniami.]{
    \label{fig:sys-with-dma}
    \includegraphics[width=.25\textwidth]{build/mmu-iommu-images--img-dma.eps}
  }\qquad
  \subfloat[Systemu z~zarówno układem MMU jak i~IOMMU, które tłumaczą
    adresy widziane przez odpowiednio procesor oraz urządzenia.]{
    \label{fig:sys-with-iommu}
    \includegraphics[width=.25\textwidth]{build/mmu-iommu-images--img-iommu.eps}
  }
  \caption[Różne przestrzenie adresowe dostępne
    w~komputerze.]{Reprezentacja systemów z~różnymi podzespołami
    uczestniczącymi w~translacji adresów lub transferach danych do
    pamięci operacyjnej.}
  \label{fig:mmu-iommu}
\end{figure}

\subsection{Jednostka translacji adresów}

Nowoczesne architektury przeznaczone do serwerów i~komputerów
osobistych posiadają jednostkę zarządzania pamięcią (\ang{memory
  management unit}, MMU), która tłumaczy adresy logiczne na fizyczne.
Dzięki temu, bufory, które z~punktu widzenia procesora są ciągłe, mogą
w~rzeczywistości być podzielone na wiele stron rozrzuconych po pamięci
fizycznej.  W~ten sposób, nawet jeżeli program alokuje
wielomegabajtowy obszar, system może zrealizować żądanie alokując
wiele czterokibibajtowych\footnote{Aby uniknąć wieloznaczności,
  stosuję przedrostki „kilo-” i~„mega-” w~znaczeniu odpowiednio tysiąc
  i~milion (zgodnie z~układem SI), a~„kibi-” i~„mebi-” w~znaczeniu
  odpowiednie $2^{10} = 1024$ i~$2^{20}$ (zgodnie
  z~tzw.\ przedrostkami ICE).} stron i~nie przejmować się fragmentacją
pamięci.

\subsection{Bezpośredni dostęp do pamięci}

Niemniej, tak jak to przedstawia rysunek \subref*{fig:sys-with-mmu},
MMU przeważnie nie jest dostępny dla pozostałych układów znajdujących
się w~urządzeniu, takich jak np.\ karta dźwiękowa, czy kontroler
sieciowy.  Także i~dla tych przypadków istnieje rozwiązanie w~postaci
mechanizmu bezpośredniego dostępu do pamięci (\ang{Direct Memory
  Access}, DMA), którego celem jest odciążenie procesora od
przesyłania danych.  Co prawda urządzenie nadal znajduje się
w~przestrzeni adresów fizycznych, co ilustruje rysunek
\subref*{fig:sys-with-dma}, ale dzięki układowi DMA nieciągłość
buforów może zostać przed nim ukryta.

Układ DMA może obsługiwać technikę wektorowego wejścia/wyjścia
(\ang{vectored I/O}), która pozwala zbierać wiele rozrzuconych
fragmentów danych w~jeden bufor (stąd też inna nazwa:
rozrzucanie/zbieranie, \ang{scatter/gather}).  Nawet jeżeli DMA nie
umożliwia wykorzystania tej techniki, procesor może ją symulować
poprzez sekwencyjne wywoływanie wielu mniejszych transferów, choć jest
to niestety mniej efektywne rozwiązanie.

Mechanizm bezpośredniego dostępu do pomięci (niezależnie czy technika
wektorowego wejścia/wyjścia jest dostępna, czy nie) jest ograniczony
do transferów sekwencyjnych.  Sprawdza się bardzo dobrze dla operacji
dyskowych, ale nie nadaje się dla sprzętowego dekodera wideo, który
potrzebuje dostępu do wielu dekodowanych ramek
jednocześnie.\footnote{Jednym z~rodzajów ramek stosowanych do
  kodowania klatki ze strumienia wideo jest b-ramka, która już nawet
  w~starszych standardach takich jak MPEG-2 może odwoływać się do
  jednej poprzedzającej i~jednej następującej klatki, a w~przypadku
  nowszego standardu H.264, może zależeć od więcej niż dwóch innych
  ramek.}

\subsection{IOMMU}

Oczywiście nie ma żadnych technologicznych przeszkód do zastosowania
jednostki translacji adresów również dla podzespołów innych niż
procesor.  Istotnie istnieją platformy sprzętowe z~tzw.\ MMU
wejścia/wyjścia (ang.\ IOMMU), który pozwala na budowanie dużych
ciągłych buforów złożonych ze stosunkowo małych stron.  W~takich
systemach, w~zasadzie nie ma (lub nie powinno być) konieczności
alokowania wielomegabajtowych buforów.

Niestety, nawet jeżeli IOMMU jest dostępne na danej platformie, jego
obecność może się wiązać z~dodatkowym kosztem wynikającym
z~nieoptymalnego kodu \autocite{bib:price-of-safety} lub konieczności
odczytywania mapowań z~pamięci
\autocite{bib:mitigate-iotlb-bottleneck}.  Z~tego względu architekt
platformy może zdecyduje się wyłączyć IOMMU i~rozwiązać problem
„w~oprogramowaniu”.

\subsection{Podsumowanie}

Z~uwagi na koszty i~ograniczenia zarówno kontrolerów DMA jak i~układów
MMU, w~wielu systemach wbudowanych, takich jak np.\ telefony
komórkowe, takie mechanizmy są często niedostępne.  Jednocześnie,
właśnie takie urządzenia posiadają dużo wyspecjalizowanych
podzespołów, jak chociażby aparat fotograficzny, czy układ do
szybkiego dekodowania obrazów JPEG.

Powoduje to, że tego typu układy muszą operować bezpośrednio na
adresach fizycznych i~w~konsekwencji, wszelkie stosowane przez nie
bufory muszą być ciągłe w~pamięci fizycznej.  Niestety, Linux nie jest
dobrze przystosowany do alokowania takich
obszarów.\footnote{W~szczególności, Linux nie jest nawet w~stanie (bez
  modyfikowania źródła) zarządzać obszarami większymi niż cztery
  mebibajty (1024 strony), gdy tymczasem pięciomegapikselowa kamera
  potrzebuje buforu o~rozmiarze 15 megabajtów, a~pojedyncza ramka {\it
    full HD} (tj.\ $1920 \times 1080$) zajmuje ponad sześć megabajtów.}


\section{Możliwe rozwiązania}

Ponieważ opisany powyżej problem jest znany od dawna, na przestrzeni
lat powstało wiele rozwiązań programowych umożliwiających obejść
trudność w~alokacji dużych obszarów.  W~tym podrozdziale opiszę je
pokrótce oraz przedstawie ich ograniczenia.

\subsection{Przypisywanie pamięci na stałe}

Najprostszym, i~stosunkowo często stosowanym, rozwiązaniem jest
rezerwacja przy starcie systemu pewnego regionu pamięci na potrzeby
konkretnych sterowników.

Najłatwiejszym, acz niezbyt eleganckim sposobem jest wykorzystanie
argumentu \code|mem| jądra.  Przekazany przez bootloader powoduje, że
Linux nie stara się automatycznie wykryć dostępnej w~systemie pamięci
RAM i~zamiast tego interpretuje przekazane informacje.  W~ten sposób,
możliwe jest ograniczenie widzianej przez system pamięci, tak że
ukryte regiony mogą być wykorzystywane przez konkretne sterowniki.

Bardziej eleganckim rozwiązaniem jest skorzystanie z~alokatora
memblock, który jest aktywny zanim jądro zainicjuje wszystkie swoje
podsystemy.  Jego zadaniem jest śledzenie wolnej pamięci zanim jeszcze
bardziej zaawansowany alokator stron będzie dostępny w~systemie.
Wołany dostatecznie wcześnie, jest w~stanie zaalokowanie duże obszary
pamięci, które potem można wykorzystać w~dowolny sposób.

Niestety, o~ile tego typu rozwiązania mogą być wystarczające, jeżeli
podzespoły wymagają stosunkowo małych buforów, przestaje się on
skalować przy współczesnych systemach, gdyż wymaga rezerwacji wielu
megabajtów pamięci, która przez większość czasu nie jest do niczego
wykorzystywana.

\subsection{Pula pamięci fizycznej}

Bardziej skomplikowanym rozwiązaniem jest mechanizm, który rezerwuje
pewną przestrzeń pamięci, ale zamiast na stałe przypisywać obszary do
urządzeń, pozwala sterownikom alokować bufory, wtedy, gdy są one
potrzebne.

W~trakcie moich prac stworzyłem {\it Physical Memory Manager} (lub
PMM) \autocite{patch:pmm}, który implementuje dokładnie te
założenia. W~tym podstawowym założeniu, PMM nie przedstawia sobą nic
nowego.  Już bowiem w~1996 roku Matt Welsh napisał pierwszą wersję
patcha \emph{bigphysarea} dla jądra 1.3.71, który był z~różnym
zaangażowaniem utrzymywany i~przystosowywany aż do wersji
3.2 Linuksa \autocite{patch:bigphys}.

PMM umożliwiał alokowanie dużych obszarów ciągłej pamięci fizycznej
nie tylko sterownikom działającym w~przestrzeni jądra, ale także
programom działającym pod kontrolą systemu.  W~ten sposób, aplikacja
mogła zaalokować pamięć, tak aby (dla przykładu) dekoder JPEG mógł
zdekodować plik graficzny i~umieścić go w~buforach bezpośrednio
dostępnych dla aplikacji.  W~ten sposób, kod odpowiedzialny za
mapowanie zaalokowanych obszarów był umieszczony wewnątrz PMM
i poszczególne sterowniki nie musiały się tym problemem zajmować.

Co więcej, PMM został zintegrowany z~mechanizmem współdzielenia
pamięci Systemu V~(tj.\ funkcjami \code|shmget|, \code|shmat|,
\code|shmdt| itp.) wykorzystywanym między innymi przez system X~Window
do umożliwienia współdzielenia map bitowych pomiędzy klientem
i~serwerem (działającymi na tej samej maszynie) bez konieczności
przysłania danych przez gniazdo sieciowe.

Pozwalało to na dekodowanie obrazów i~strumieni video bezpośrednio do
buforów z~których X11 odczytywał dane.  Dzięki temu, w~całym procesie
dane nie były niepotrzebnie kopiowane z~jednego miejsca w~pamięci do
drugiego, co minimalizowało użycie procesora i~szyny pamięci
i~w~rezultacie przyśpieszało działanie systemu.

Jednakże, pamięć zarezerwowana przez PMM i~tak przez większość czasu
była zupełnie nieużywana, a~więc marnowana.  Z~tego powodu, PMM
w~formie zaprezentowanej na początku nie został przyjęty przez
społeczność programistów Linuksa i~musiałem rozwijać inne rozwiązanie.

\subsection{Zarys Contiguous Memory Allocatora}

Aby rozwiązać i~ten problem, {\it Contiguous Memory Allocator}
umożliwia systemowi używanie zarezerwowanej pamięci, o~ile żadne
urządzenie jej w~danym momencie nie potrzebuje.

W~swoich początkowych wersjach, również mechanizm CMA działał na
założeniach podobnych do PMM -- rezerwował przy starcie systemu
pamięć, którą potem zarządzał pozwalając sterownikom i~programom na
alokowanie obszarów ciągłych fizycznie \autocite{patch:cma-4}.

Wynika to z~faktu, iż pierwsze wersje alokatora CMA skupiały się
w~dużej mierze na rozwiązywaniu problemu przypisywania różnych
zarezerwowanych obszarów do różnych urządzeń, a~także umożliwianiu
sterownikom alokowanie różnych buforów w~różnych obszarach pamięci.
Było to potrzebne, gdyż dekoder wideo (\ang{Multi-Format Codec}, lub
MFC) stosowany na platformie S5PV110 wymagał, aby różne dane były
przechowywane w~różnych bankach pamięci.  Pozwalało to na zwiększenie
szybkości dostępu do tych danych dzięki zastosowaniu odczytu z~dwóch
banków pamięci jednocześnie.

Z~czasem, coraz bardziej integrowałem mechanizm CMA z~kodem
zarządzania pamięci w~Linuksie w~wyniku czego, pamięć rezerwowana przy
starcie systemu, stała się dostępna dla reszty systemu, o~ile żaden
sterownik jej nie używał \autocite{patch:cma-24}.  Takie rozwiązanie
zostało ostatecznie zaakceptowane przez społeczność deweloperów
Linuksa i~jest dostępne w~źródłach Linuksa począwszy od wersji 3.5.
W~tej pracy opisuję alokator CMA w~formie w~jakiej znalazł się on
w~Linuksie 3.5\footnote{Należy zauważyć, że Linux jest szybko
  rozwijającym się projektem wolnego oprogramowania i~ponieważ
  mechanizm CMA używana jest przez coraz więcej osób, jest ona ciągle
  rozwijana i~im dalej w~przyszłość, tym bardziej opis w~niniejszej
  pracy będzie się różnił od stanu faktycznego.}.

\section{Wielkie strony}

Zagadnieniem związanym w~pewnym stopniu z~mechanizmem CMA są wielkie
strony (\ang{huge pages}).  O~ile zwyczajne strony pamięci mają
przeważnie cztery kibibajty, o~tyle wielkie strony mają rozmiary rzędu
dwóch lub czterech mebibajtów\footnote{Konkretne rozmiary zależą od
  architektury procesora i~co więcej wiele rozmiarów może być
  dostępnych.}.  Stosowane są w~celu zmniejszenia liczby wpisów
w~tablicy translacji adresów, a~co za tym idzie również TLB procesora.

Począwszy od wersji 2.6.38, Linux posiada mechanizm automatycznego
wykorzystywania wielkich stron dla działających programów,
\ang{transparent huge pages} \autocite{bib:v2.6.38}.  Dzięki niemu,
o~ile to możliwe, wiele czterokibibajtowych stron mapowanych jest za
pomocą pojedynczego wpisu w~tablicy translacji adresów.

Podobnie jak w~przypadku mechanizmu CMA, wymaga to alokowania dużych
obszarów ciągłych fizycznie.  Tym co różni oba mechanizmy jest wymóg
aby alokacja CMA zakończyła się sukcesem i~do tego w~jak najkrótszym
czasie, gdy tymczasem mechanizm automatycznego wykorzystania wielkich
stron jest oportunistyczny i~jeżeli w~danej chwili w~systemie nie ma
dostatecznie dużego wolnego obszaru, mechanizm ten nie jest
wykorzystywany.

Z~uwagi na te odmienne wymagania, obie implementacje, pomimo, że
pozornie mające podobne założenia, są w~dużym stopniu rozłączne.
