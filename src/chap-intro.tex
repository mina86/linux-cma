\chapter{Wstęp}

Tematem niniejszej pracy inżynierskiej jest sterownik dla jądra Linux,
który pozwala w~efektywny sposób alokować duże obszary ciągłej
fizycznie pamięci.  Opisanym mechanizmem jest stworzony przeze mnie
alokator ciągłej pamięci, (\ang{Contiguous Memory Allocator}, CMA).

Podstawowym zastosowaniem dużych buforów, który w~głównej mierze
brałem podczas pisania alokatora CMA, jest wykorzystanie ich
w~podzespołach dostępnych w~nowoczesnych telefonach komórkowych.
Niemniej odkąd stworzony przeze mnie kod został dołączony do Linuksa,
różne osoby wykazały zainteresowanie, aby wykorzystywać go również
w~innych celach.


\section{Opis problemu}

W~celu zwiększenia efektywności działania oraz liczby udostępnianych
funkcji, komputery a w~szczególności telefony komórkowe posiadają
wiele wyspecjalizowanych podzespołów.  W~wielu przypadkach, procesor
komunikuje się z~nimi poprzez bufory w pamięci operacyjnej,
przekazując do urządzenia jedynie adresy gdzie dane się znajdują.
Dostęp do RAM-u poprzez takie podzespoły może się jednak wiązać
z~wieloma ograniczeniami.

\begin{figure}[tbp]
  \centering
  \subfloat[System z~układem MMU pomiędzy procesorem a~pamięcią oraz
    urządzeniami podłączonymi bezpośrednie do szyny pamięci.]{
    \label{fig:sys-with-mmu}
    \includegraphics[width=.25\textwidth]{build/mmu-iommu-images--img-nommu.eps}
  } \qquad
  \subfloat[Systemu z~kontrolerem DMA, który pośredniczy w~transferach
    danych pomiędzy pamięcią i~urządzeniami.]{
    \label{fig:sys-with-dma}
    \includegraphics[width=.25\textwidth]{build/mmu-iommu-images--img-dma.eps}
  }\qquad
  \subfloat[Systemu z~zarówno układem MMU jak i~IOMMU, które tłumaczą
    adresy widziane przez odpowiednio procesor oraz urządzenia.]{
    \label{fig:sys-with-iommu}
    \includegraphics[width=.25\textwidth]{build/mmu-iommu-images--img-iommu.eps}
  }
  \caption[Różne przestrzenie adresowe dostępne
    w~komputerze.]{Reprezentacja systemów z~różnymi podzespołami
    uczestniczącymi w~translacji adresów lub transferach danych do
    pamięci operacyjnej.}
  \label{fig:mmu-iommu}
\end{figure}

\subsection{Jednostka translacji adresów}

Nowoczesne architektury przeznaczone do serwerów i~komputerów
osobistych posiadają jednostkę zarządzania pamięcią (\ang{memory
  management unit}, MMU), która tłumaczy adresy logiczne na
fizyczne.  Dzięki temu, bufory, które z~punktu widzenia procesora są
ciągłe, mogą w~rzeczywistości być podzielone na wiele stron
rozrzuconych po pamięci fizycznej.  W~ten sposób, nawet jeżeli program
alokuje wielomegabajtowy obszar, system może zrealizować żądanie
alokując wiele czterokibibajtowych\footnote{Aby uniknąć
  wieloznaczności, stosuję przedrostki „kilo-” i~„mega-” w~znaczeniu
  odpowiednio tysiąc i~milion (zgodnie z~układem SI), a~„kibi-”
  i~„mebi-” w~znaczeniu odpowiednie $2^{10} = 1024$ i~$2^{20}$
  (zgodnie z~tzw.\ przedrostkami ICE).} stron i~nie przejmować się
fragmentacją pamięci.

\subsection{Bezpośredni dostęp do pamięci}

Niemniej, tak jak to przedstawia rysunek \subref*{fig:sys-with-mmu},
jednostka MMU przeważnie nie jest dostępna dla pozostałych układów
znajdujących się w~urządzeniu, takich jak np.\ karta dźwiękowa, czy
kontroler sieciowy.  Na szczęście także i~dla tych przypadków istnieje
rozwiązanie w~postaci mechanizmu bezpośredniego dostępu do pamięci
(\ang{Direct Memory Access}, DMA), którego celem jest odciążenie
procesora od przesyłania danych.  Co prawda urządzenie nadal znajduje
się w~przestrzeni adresów fizycznych, co ilustruje rysunek
\subref*{fig:sys-with-dma}, ale dzięki układowi DMA nieciągłość
buforów może zostać przed nim ukryta.

Układ DMA może obsługiwać technikę wektorowego wejścia/wyjścia
(\ang{vectored I/O}), która pozwala zbierać wiele rozrzuconych
fragmentów danych w~jeden bufor (stąd też inna nazwa:
rozrzucanie/zbieranie, \ang{scatter/gather}).  Nawet jeżeli DMA nie
umożliwia wykorzystania tej techniki, procesor może ją symulować
poprzez sekwencyjne wywoływanie wielu mniejszych transferów, choć jest
to niestety mniej efektywne rozwiązanie.

Mechanizm bezpośredniego dostępu do pomięci jest jednak ograniczony do
transferów sekwencyjnych.  Sprawdza się bardzo dobrze dla operacji
dyskowych, ale nie nadaje się dla sprzętowego dekodera wideo, który
potrzebuje dostępu do wielu dekodowanych ramek
jednocześnie.\footnote{Jednym z~rodzajów ramek stosowanych do
  kodowania klatki ze strumienia wideo jest b-ramka, która już nawet
  w~starszych standardach takich jak MPEG-2 może odwoływać się do
  jednej poprzedzającej i~jednej następującej klatki, a w~przypadku
  nowszego standardu H.264, może zależeć od więcej niż dwóch innych
  ramek.}

\subsection{IOMMU}

Oczywiście nie ma żadnych technologicznych przeszkód do zastosowania
jednostki translacji adresów również dla podzespołów innych niż
procesor.  Istotnie istnieją platformy sprzętowe z~tzw.\ MMU
wejścia/wyjścia (IOMMU), który pozwala na budowanie dużych ciągłych
buforów złożonych ze stosunkowo małych stron.  W~takich systemach
w~zasadzie nie ma (lub nie powinno być) konieczności alokowania
wielomegabajtowych buforów.

Niestety, nawet jeżeli układ IOMMU jest dostępny, jego obecność może
się wiązać z~dodatkowym kosztem wynikającym z~nieoptymalnego kodu
(\textcite{bib:price-of-safety} pokazuje zwiększenie zużycia procesora
wynikające z~nieoptymalnego mapowania o~15--30\%) lub konieczności
odczytywania mapowań z~pamięci
(\textcite{bib:mitigate-iotlb-bottleneck} pokazuje, że przy
nieoptymalnych odczytach, czas dostępu do pamięci w~buforach rzędu
\unit[1]{MiB} może wynieść nawet 45\%).  Z~tego względu architekt
platformy może zdecydować się wyłączyć IOMMU i~rozwiązać problem
„w~oprogramowaniu”.

\subsection{Podsumowanie}

Z~uwagi na koszty i~ograniczenia zarówno kontrolerów DMA jak i~układów
MMU, w~wielu systemach wbudowanych, takich jak np.\ telefony
komórkowe, takie mechanizmy są często niedostępne.  Jednocześnie,
właśnie takie urządzenia posiadają dużo wyspecjalizowanych
podzespołów, jak chociażby aparat fotograficzny, czy układ do
szybkiego dekodowania obrazów JPEG.

Powoduje to, że tego typu układy muszą operować bezpośrednio na
adresach fizycznych i~w~konsekwencji, wszelkie stosowane przez nie
bufory muszą być ciągłe w~pamięci fizycznej.  Niestety, Linux nie jest
dobrze przystosowany do alokowania takich
obszarów.\footnote{W~szczególności, Linux nie jest nawet w~stanie (bez
  modyfikowania źródła) zarządzać obszarami większymi niż cztery
  mebibajty (1024 strony), gdy tymczasem pięciomegapikselowa kamera
  potrzebuje buforu o~rozmiarze 15 megabajtów, a~pojedyncza ramka
  \ang*{full HD} (tj.\ 1920 $\times$ 1080) zajmuje ponad sześć
  megabajtów.}


\section{Możliwe rozwiązania}

Ponieważ opisany powyżej problem jest znany od dawna, na przestrzeni
lat powstało wiele rozwiązań programowych umożliwiających obejście
trudność w~alokacji dużych obszarów.  W~tym podrozdziale opiszę je
pokrótce oraz przedstawie ich ograniczenia.

\subsection{Przypisywanie pamięci na stałe}

Najprostszym, i~stosunkowo często stosowanym, rozwiązaniem jest
rezerwacja przy starcie systemu pewnego regionu pamięci na potrzeby
konkretnych sterowników.

Najłatwiejszym, acz niezbyt eleganckim sposobem jest wykorzystanie
argumentu \code|mem| jądra.  Przekazany przez program rozruchowy
powoduje, że Linux nie stara się automatycznie wykryć dostępnej
w~systemie pamięci RAM i~zamiast tego interpretuje przekazane
informacje.  W~ten sposób, możliwe jest ograniczenie widzianej przez
system pamięci, tak że ukryte regiony mogą być wykorzystywane przez
konkretne sterowniki.

Bardziej eleganckim rozwiązaniem jest skorzystanie z~alokatora
memblock, który jest aktywny zanim jądro zainicjuje wszystkie swoje
podsystemy.  Jego zadaniem jest śledzenie wolnej pamięci zanim jeszcze
bardziej zaawansowany alokator stron będzie dostępny w~systemie.
Wołany dostatecznie wcześnie, jest w~stanie zaalokować duże obszary
pamięci, które potem można wykorzystać w~dowolny sposób.

Niestety, o~ile tego typu rozwiązania mogą być wystarczające, jeżeli
podzespoły wymagają stosunkowo małych buforów lub jeżeli pamięć jest
cały czas wykorzystywana (np.\ w~przypadku bufora ramki,
\ang{framebuffor}), przestaje się on skalować przy współczesnych
systemach, gdyż wymaga rezerwacji wielu megabajtów pamięci, która
przez większość czasu nie jest do niczego wykorzystywana.

\subsection{Pula pamięci fizycznej}\label{sec:intro-pmm}

Bardziej skomplikowanym rozwiązaniem jest mechanizm, który rezerwuje
pewną przestrzeń pamięci, ale zamiast na stałe przypisywać obszary do
urządzeń, pozwala sterownikom alokować bufory, wtedy, gdy są one
potrzebne.

W~trakcie moich prac stworzyłem \ang*{Physical Memory Manager} (PMM)
\autocite{patch:pmm}, który implementuje dokładnie te założenia. W~tym
podstawowym zastosowaniu, PMM nie przedstawia sobą nic nowego.  Już
bowiem w~1996 roku Matt Welsh napisał pierwszą wersję dodatku
\ang*{bigphysarea} dla jądra 1.3.71, który był z~różnym zaangażowaniem
utrzymywany i~przystosowywany aż do wersji 3.2 Linuksa
\autocite{patch:bigphys}.

PMM posiadał jednak wiele dodatkowych funkcji opisanych w~podrozdziale
\ref{sec:evo-pmm}, których próżno szukać w~\ang*{bigphysarea}, czy
w~innych dostępnych poprawkach do jądra.  Niemniej, pomimo swoich
dodatkowych funkcji, nie rozwiązywał do końca problemu nieefektywnego
wykorzystania pamięci, przez co nie został przyjęty przez społeczność
programistów Linuksa i~musiałem rozwijać inne rozwiązanie.

\subsection{Zarys Contiguous Memory Allocatora}

Ostatecznym rozwiązaniem stał się alokator ciągłej pamięci (CMA),
który umożliwia systemowi używanie zarezerwowanej pamięci, o~ile żadne
urządzenie jej w~danym momencie nie potrzebuje.

W~swoich początkowych wersjach, również mechanizm CMA działał na
założeniach podobnych do PMM\,---\,rezerwował przy starcie systemu
pamięć, którą potem zarządzał pozwalając sterownikom i~programom na
alokowanie obszarów ciągłych fizycznie \autocite{patch:cma-1}.

Wynika to z~faktu, iż pierwsze wersje alokatora CMA skupiały się
w~dużej mierze na rozwiązywaniu problemu przypisywania różnych
zarezerwowanych obszarów do różnych urządzeń, a~także umożliwianiu
sterownikom alokowanie różnych buforów w~różnych obszarach pamięci, co
opisałem w~podrozdziale \ref{sec:evo-cma}.

Z~czasem, coraz bardziej integrowałem mechanizm CMA z~kodem
zarządzania pamięci w~Linuksie w~wyniku czego, pamięć rezerwowana przy
starcie systemu, stała się dostępna dla reszty systemu, o~ile żaden
sterownik jej nie używał \autocite{patch:cma-24}.  Takie rozwiązanie
zostało ostatecznie zaakceptowane przez społeczność deweloperów
Linuksa i~jest dostępne w~źródłach Linuksa począwszy od wersji 3.5.
W~tej pracy opisuję alokator CMA w~formie w~jakiej znalazł się on
w~Linuksie 3.5\footnote{Należy zauważyć, że Linux jest szybko
  rozwijającym się projektem wolnego oprogramowania i~ponieważ
  mechanizm CMA używana jest przez coraz więcej osób, jest ona ciągle
  rozwijana i~im dalej w~przyszłość, tym bardziej opis w~niniejszej
  pracy będzie się różnił od stanu faktycznego.}.

\section{Wielkie strony}

Zagadnieniem związanym w~pewnym stopniu z~mechanizmem CMA są wielkie
strony (\ang{huge pages}).  O~ile zwyczajne strony pamięci mają
przeważnie cztery kibibajty, o~tyle wielkie strony mają rozmiary rzędu
dwóch lub czterech mebibajtów\footnote{Konkretne rozmiary zależą od
  architektury procesora i~co więcej wiele rozmiarów może być
  dostępnych jednocześnie.}.  Stosowane są w~celu zmniejszenia liczby
wpisów w~tablicy translacji adresów, a~co za tym idzie również TLB
procesora.

Począwszy od wersji 2.6.38, Linux posiada mechanizm automatycznego
wykorzystywania wielkich stron dla działających programów,
\ang{transparent huge pages} \autocite{bib:v2.6.38}.  Dzięki niemu,
o~ile to możliwe, wiele czterokibibajtowych stron mapowanych jest za
pomocą pojedynczego wpisu w~tablicy translacji adresów.

Podobnie jak w~przypadku alokatora CMA, wymaga to alokowania dużych
obszarów ciągłych fizycznie.  Tym co różni oba mechanizmy jest wymóg
aby alokacja CMA zakończyła się sukcesem i~do tego w~jak najkrótszym
czasie, gdy tymczasem automatyczne wykorzystywanie wielkich stron jest
procesem oportunistyczny i~jeżeli w~danej chwili w~systemie nie ma
dostatecznie dużego wolnego obszaru, mechanizm ten nie zostanie
wykorzystywany.

Z~uwagi na te odmienne wymagania, obie implementacje, pomimo, że
pozornie mające podobne założenia, są w~dużym stopniu rozłączne.
