\documentclass[a4paper,11pt]{article}
\usepackage[MeX]{polski}
\usepackage[utf8]{inputenc}
\usepackage{url}
\usepackage{graphicx}
\usepackage{multicol}
\usepackage[hmargin=1in,vmargin=1in]{geometry}

\renewcommand{\thesection}{\arabic{section}.}
\renewcommand{\thesubsection}{\thesection\arabic{subsection}.}
\renewcommand{\thefigure}{\arabic{figure}.}

\title{Alokacja ciągłych fizycznie obszarów pamięci w~systemie Linux}
\author{Michał Nazarewicz \\ {\small Instytut Informatyki}}


\begin{document}

\twocolumn[\maketitle\begin{@twocolumnfalse}

\section*{Abstrakt}

Niniejszy dokument opisuje sposób alokacji dużych obszarów ciągłej
fizycznie pamięci w~systemach opartych na jądrze Linux.  Zastosowanie
w~procesorach jednostek zarządzania pamięcią pozwala na uniknięcie
konieczności takich alokacji, jednak w~systemach wbudowanych
poszczególne komponenty są często pozbawiony jednostki transakcji
adresów pamięci, przez co muszą operować bezpośrednio na adresach
fizycznych.

\vspace{1cm}

\end{@twocolumnfalse}]


\end{document}
