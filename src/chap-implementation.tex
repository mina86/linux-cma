\chapter{Implementacja i~sposób działania mechanizmu \acc{CMA}}\label{sec:implementation}

Podstawowym założeniem alokatora \acc{CMA} jest umożliwienie alokowania
dużych obszarów ciągłych fizycznie bez konieczneści rezerwacji na
wyłączność dużej ilości pamięci.  Aby to umożliwić, interfejs \acc{CMA}
korzysta z~mechanizmu migracji stron opisanego pokrótce w~podrozdziale
\ref{sec:migratetype}.  Ogólny zarys alokacji z~regionów \acc{CMA}
przedstawiony jest na rysunku \ref{fig:cma-alloc-algo} a~niniejszy
rozdział opisze ją bardziej szczegółowo.

\begin{figure}[tbp]
  \includegraphics[width=\textwidth]{build/cma-alloc-algo.eps}
  \caption{Schemat działania alokatora \acc{CMA}.}
  \label{fig:cma-alloc-algo}
\end{figure}

\section{Typ migracji \acc{CMA}}\label{sec:migrate-cma}

Migracja jest możliwa tylko dla stron ruchomych.  Niestety, przed
zaimplementowaniem alokatora \acc{CMA}, Linux nie posiadał mechanizmu,
który pozwalałby zagwarantować, aby w~systemie istniał duży obszar,
w~którym strony są albo wolne, albo ruchome.  Ponieważ (jak opisałem
w~podrozdziale \ref{sec:type-change}) jądro dopuszcza alokacje
nieruchomych stron z~bloków stron oznaczonych jako przechowujące
strony ruchome, a~także posiada mechanizm na skutek którego bloki
stron zmieniają swój typ, oznaczenie bloku stron jako ruchome nie
gwarantuje, że tylko strony ruchome będą alokowane z~tego bloku stron.

Z~tego powodu, aby mechanizm \acc{CMA} działał dobrze, pierwszym
krokiem było stworzenie nowego typu migracji, nazwanego po prostu
typem migracji \acc{CMA} (\code|MIGRATE_CMA|), który posiada dwie
bardzo istotne cechy:

\begin{itemize}
\item Z~bloków stron oznaczonych typem \acc{CMA} mogą być alokowane tylko
  strony ruchome.  Innymi słowy, typ migracji \acc{CMA} istnieje w~liście
  zapasowych typów migracji jedynie dla typu ruchomego.
\item Blok oznaczony typem \acc{CMA} nie zmienia swojego typu (na skutek
  działania alokatora stron).
\end{itemize}

O~ile pierwsza właściwość jest stosunkowo prosta do osiągnięcia,
zagwarantowanie niezmienności typu bloku stron wymagało
zidentyfikowania wszystkich sytuacji, w~których blok może zmienić swój
typ i~dodania odpowiednich warunków zapewniających, że niepożądana
zmiana nie nastąpi.

\section{Alokowanie wybranego obszaru pamięci}\label{sec:alloc-contig-range}

W~sytuacji, gdy istnieje gwarancja, że dany zakres stron posiada
jedynie strony wolne i~ruchome, można przystąpić do jej alokacji.
Drugim krokiem implementowania alokatora \acc{CMA} było stworzenie funkcji,
która dostaje jako argument zakres stron, a~następnie migruje
wszystkie zajęte strony, a~wolne usuwa z~listy wolnych stron.

Wydruk \ref{lst:alloc-contig-range} przedstawia skrócony (bo
pozbawiony sprawdzania błędów oraz komentarzy) kod źródłowy funkcji
\code|alloc_contig_range| znajdującej się w~Linuksie 3.5.  Jest ona
bardziej skomplikowana, niżby to wynikało z~powyższego opisu.

\begin{lstlisting}[float=tb,caption=Skrócony wydruk funkcji
    \code|alloc_conting_range| z Linuksa 3.5.,label=lst:alloc-contig-range]
int alloc_contig_range(unsigned long start, unsigned long end,
                       unsigned migratetype)
{
	struct zone *zone = page_zone(pfn_to_page(start));
	unsigned long outer_start, outer_end;
	int ret = 0, order;

	start_isolate_page_range(pfn_max_align_down(start),
	                         pfn_max_align_up(end), migratetype);

	__alloc_contig_migrate_range(start, end);

	lru_add_drain_all();
	drain_all_pages();

	order = 0;
	outer_start = start;
	while (!PageBuddy(pfn_to_page(outer_start)))
	        outer_start &= ~0UL << ++order;

	outer_end = isolate_freepages_range(outer_start, end);

	if (start != outer_start)
		free_contig_range(outer_start, start - outer_start);
	if (end != outer_end)
		free_contig_range(end, outer_end - end);

	undo_isolate_page_range(pfn_max_align_down(start),
	                        pfn_max_align_up(end), migratetype);
	return ret;
}
\end{lstlisting}

Pierwszym krokiem jest wywołanie funkcji
\code|start_isolate_page_range|, której celem jest zmiana typu bloków
stron na izolowany typ migracji i~przeniesienie wszystkich wolnych
stron należących do tych bloków na listy tego typu.  Pomimo, że
izolowane strony są przechowywane na liście wolnych stron, i~formalnie
są pod kontrolą alokatora stron, nie są nigdy używane do zaspokajania
żądań alokacji.  W~ten sposób, funkcja \code|alloc_contig_range|
uzyskuje pewność, że w~trakcie jej działania strony, na których
operuje nie zostaną zaalokowane dla innych wątków jądra.

W~dalszej części wołana jest funkcja
\code|__alloc_contig_migrate_range|, której zadaniem jest
zidentyfikowanie i~zmigrowanie wszystkich zajętych stron z~zakresu
$\left<\mathrm{start}, \mathrm{end}\right)$.  Również i~tę funkcję
musiałem zaimplementować specjalnie na potrzeby mechanizmu \acc{CMA} i~jest
ona opisana w~następnym podrozdziale.

Po tych dwóch krokach w~zasadzie wszystkie strony z~interesującego nas
zakresu powinny być wolne.  Jednak, jak opisano w~podrozdziale
\ref{sec:pcp-lists}, strona może być wolna, ale nie znajdować się na
liście wolnych stron.  Dlatego, w~dalszej części, w~liniach 13--14
wszystkie strony przenoszone są z~list \acc{PCP} i~wstawiane z~powrotem na
listy wolnych stron.

Po tych operacjach wszystkie strony w~zakresie $\left<\mathrm{start},
\mathrm{end}\right)$ są wolne, jednak z~uwagi na sposób działania
algorytmu bliźniaków, strona identyfikowana przez \code|start| nie
musi wcale być poprawną wolną stroną.  Wynika to z~faktu, że
w~procesie zwalniania stron (linia \ref{alg:buddy-free:join} algorytmu
\ref{alg:buddy-free}) mogła ona zostać połączona ze swoją bliźniaczą
stroną tworząc wolną stronę o~wyższym rzędzie.  Z~tego powodu,
w~liniach 16--19 funkcja \code|alloc_contig_range| wyszukuje tę
początkową stronę\,---\,„zewnętrzny początek” zakresu.

W~linii 21 następuje właściwe zaalokowanie stron, tj.\ usunięcie ich
z~list wolnych stron, na skutek czego alokator stron zupełnie nie
zdaje sobie sprawy z~ich istnienia.  W~procesie tym, wszystkie strony
których rząd jest niezerowy są dzielone na strony rzędu zerowego.

Należy zauważyć, że tak samo jak na początku zakresu, tak samo i~na
końcu możemy trafić na stronę, która przekracza interesujący nas
zakres.  Dlatego też funkcja \code|isolate_freepages_range|, która
dokonuje alokacji, zwraca „zewnętrzny koniec” zakresu, który może
wypadać poza żądanym końcem zakresu.

Po tych wszystkich operacjach, funkcja zaalokowała strony z~zakresu
$\left<\mathrm{outer\_start}, \mathrm{outer\_end}\right)$, który może
być większy od żądanego $\left<\mathrm{start}, \mathrm{end}\right)$.
Niepotrzebne strony są zwracane do alokatora stron w~liniach 23--26.

Ostatnim krokiem jest przywrócenie pierwotnego typu migracji blokom
stron, na których funkcja operowała, co jest dokonane w~linii 28.


\section{Migracja zakresu stron}\label{sec:alloc-contig-migrate-range}


Wydruk \ref{lst:alloc-contig-migrate-range} przedstawia funkcję
\code|__alloc_contig_migrate_range|, której zadaniem jest zmigrowanie
zakresu stron.  Funkcja działa w~pętli dopóki wszystkie zajęte strony
z~zakresu $\left<\mathrm{start}, \mathrm{end}\right)$ nie zostaną
zwolnione.


\begin{lstlisting}[float=tb,caption={Skrócony wydruk funkcji
    \code|__alloc_conting_migrate_range| z Linuksa 3.5.},label=lst:alloc-contig-migrate-range]
static int __alloc_contig_migrate_range(unsigned long start, unsigned long end)
{
	unsigned long pfn = start;
	unsigned int tries = 0;
	int ret = 0;

	struct compact_control cc = {
		.nr_migratepages = 0,
		.order = -1,
		.zone = page_zone(pfn_to_page(start)),
		.sync = true,
	};
	INIT_LIST_HEAD(&cc.migratepages);

	migrate_prep_local();

	while (pfn < end || !list_empty(&cc.migratepages)) {
		if (list_empty(&cc.migratepages)) {
			cc.nr_migratepages = 0;
			pfn = isolate_migratepages_range(cc.zone, &cc,
			                                 pfn, end);
			tries = 0;
		} else if (++tries == 5) {
			ret = ret < 0 ? ret : -EBUSY;
			break;
		}

		ret = migrate_pages(&cc.migratepages,
		                    __alloc_contig_migrate_alloc,
		                    0, false, MIGRATE_SYNC);
	}

	putback_lru_pages(&cc.migratepages);
	return ret > 0 ? 0 : ret;
}
\end{lstlisting}

Kolejka stron do zmigrowania przechowywana jest na liście
\code|cc.migratepages|, która (jeżeli jest pusta) jest uzupełniana
przez funkcję \code|isolate_migratepages_range|, która skanuje podany
zakres szukając stron, które mogą zostać zmigrowane, aż znajdzie 32
strony (ten arbitralny bądź co bądź limit, zdefiniowany jest przez
stałą \code|COMPACT_CLUSTER_MAX|) lub dojdzie do końca zakresu.

Strony są migrowane przez funkcję \code|migrate_pages| i~jeżeli
wszystko się powiedzie, funkcja \code|__alloc_contig_migrate_range|
kończy się powodzeniem po przeniesieniu wszystkich zajętych stron do
innego obszaru.


\section{Wybór zakresu stron}\label{sec:alloc-from-contig}

Funkcja \code|alloc_contig_range| potrafi zaalokować zakres stron
(wykonując przy tym migrację zajętych stron), ale w~jaki sposób
mechanizm \acc{CMA} wybiera zakres do wykonania alokacji?  Odpowiedzi
na to pytanie należy szukać w~funkcji
\code|dma_alloc_from_contiguous|, której skróconą wersję przedstawia
wydruk \ref{lst:dma-alloc-from-contiguous}.

\begin{lstlisting}[float=tb,caption={Skrócony wydruk funkcji
    \code|dma_alloc_from_contiguous| z Linuksa
    3.5.},label=lst:dma-alloc-from-contiguous]
struct page *dma_alloc_from_contiguous(struct device *dev, int count,
				       unsigned int align)
{
	unsigned long mask, pfn, pageno, start = 0;
	struct cma *cma = dev_get_cma_area(dev);

	mask = (1 << align) - 1;
	for (;;) {
		pageno = bitmap_find_next_zero_area(cma->bitmap, cma->count,
						    start, count, mask);
		if (pageno >= cma->count)
			return NULL;
		pfn = cma->base_pfn + pageno;
		if (alloc_contig_range(pfn, pfn + count, MIGRATE_CMA) == 0) {
			bitmap_set(cma->bitmap, pageno, count);
			return pfn_to_page(pfn);
		}
		start = pageno + mask + 1;
	}
}
\end{lstlisting}

Funkcja używa metody „pierwszy pasujący” do wyszukania pasującego
obszaru w~bitmapie (linie 9--10).  Po wybraniu obszaru, wołana jest
funkcja \code|alloc_contig_range|, aby dany obszar zaalokować
(linia 14) i~jeżeli się to powiedzie, oznacza obszar w~bitmapie jako
zajęty i~zwraca wynik (linie 15--16).

Aby nie załączać zbyt wielu szczegółów, podrozdział
\ref{sec:alloc-contig-range} nie opisuje sytuacji, w~których alokacja
może się nie powieść, ale takie istnieją\footnote{W~istocie, jest ich
  dość sporo i~obecnie wraz z~innymi deweloperami Linuksa staram się
  wyszukać i~wyeliminować takie sytuacje.  Linux, a~szczególnie
  zarządzanie pamięcią w~Linuksie, jest jednak skomplikowany i~czasem
  trudno prześledzić wszystkie zależności i~interakcje pomiędzy
  komponentami, które mogą prowadzić do błędu alokacji.} i~z~tego
powodu, funkcja \code|dma_alloc_from_contiguous| działa w~pętli w~ten
sposób, że jeżeli alokacja jednego obszaru się nie powiedzie,
w~następnej iteracji funkcja próbuje zaalokować kolejny.

\section{Regiony \acc{CMA}}

Regiony rezerwowane są przy starcie systemu wewnątrz funkcji
\code|dma_declare_contiguous| zanim jeszcze alokator stron zostanie
zainicjowany, jak to opisuje podrozdział
\ref{sec:integrate-with-arch}.  Zadaniem funkcji
\code|dma_declare_contiguous| przedstawionej w~wydruku
\ref{lst:dma-declare-contiguous} jest żądanie pamięci z~alokatora
czasu startu systemu, który nosi nazwę \ang*{memblock}.

\begin{lstlisting}[float=tb,caption={Skrócony wydruk funkcji
    \code|dma_declare_contiguous| z Linuksa
    3.5.},label=lst:dma-declare-contiguous]
int __init dma_declare_contiguous(struct device *dev, unsigned long size,
				  phys_addr_t base, phys_addr_t limit)
{
	struct cma_reserved *r = &cma_reserved[cma_reserved_count];
	unsigned long alignment;

	alignment = PAGE_SIZE << max(MAX_ORDER - 1, pageblock_order);
	base = ALIGN(base, alignment);
	size = ALIGN(size, alignment);
	limit &= ~(alignment - 1);

	if (base) {
		if (memblock_is_region_reserved(base, size) ||
		    memblock_reserve(base, size) < 0)
			return -EBUSY;
	} else {
		phys_addr_t addr = __memblock_alloc_base(size, alignment, limit);
		if (!addr)
			return -ENOMEM;
		base = addr;
	}

	r->start = base;
	r->size = size;
	r->dev = dev;
	cma_reserved_count++;

	dma_contiguous_early_fixup(base, size);
	return 0;
err:
	return base;
}
\end{lstlisting}

Jeżeli alokacja się powiedzie, informacja o~zarezerwowanym obszarze
jest zapisywana w~tablicy \code|cma_reserved|.  Jest ona odczytywana
dopiero w~późniejszej fazie startu systemu przez funkcję
\code|cma_init_reserved_areas|.  Dla każdego zarezerwowanego regionu
wywołuje funkcję \code|cma_create_area|, która tworzy struktury
\code|cma| reprezentujące pojedynczy region pamięci, kontrolowane
przez alokator \acc{CMA}.  Struktura \code|cma| posiada następujące pola:

\begin{tabular}{lll}
\code|unsigned long|   & \textbf{\code|base_pfn|} & Identyfikator początkowej strony w~regionie. \\
\code|unsigned long|   & \textbf{\code|count|} & Liczba strony w~regionie. \\
\code|unsigned long *| & \textbf{\code|bitmap|} & Bitmapa zajętych stron. \\
\end{tabular}

Pierwsze dwa identyfikują obszar w~pamięci fizycznej, gdzie znajduje
się kontekst \acc{CMA}, a~ostatnia jest bitmapą wykorzystywaną przez funkcję
\code|dma_alloc_from_contiguous| zgodnie z~opisem w~poprzednim
podrozdziale.


\section{Podsumowanie}

Z~uwagi na złożoność mechanizmu \acc{CMA}, niniejszy rozdział skupiał
się jedynie na jego najistotniejszych aspektach.  Aby prześledzić
implementację alokatora \acc{CMA} w~większych szczegółach, czytelnik
będzie musiał zwrócić się do \autocite{patch:cma-24} lub przyjrzeć się
plikom
\code|drivers/base/dma-contiguous.c|\footnote{\url{http://lxr.linux.no/linux/drivers/base/dma-contiguous.c}},
\code|mm/page_alloc.c|\footnote{\url{http://lxr.linux.no/linux/mm/page\_alloc.c}},
\code|mm/compaction.c|\footnote{\url{http://lxr.linux.no/linux/mm/compaction.c}}
oraz
\code|mm/page_isolation.c|\footnote{\url{http://lxr.linux.no/linux/mm/page\_isolation.c}}
w~jądrze Linux.  Niemniej wiedza zawarta w~tym rozdziale powinna
wystarczyć, aby móc z~łatwością prześledzić kod stojący za mechanizmem
\acc{CMA}.
