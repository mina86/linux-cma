\chapter{Dalsze prace}

W~skomplikowanych systemach, nawet jeżeli jakiś istniejący problem
zostanie rozwiązany, przeważnie jest znaleźć wiele aspektów, które
można ulepszyć.  Nie inaczej jest z~alokatorem CMA, którego dodanie
w~czerwcu 2012 r.\ do Linuksa 3.5, w~żadnym stopniu nie oznacza
zakończenia nad nim prac.

\section{Zmiany wprowadzone po wydaniu Linuksa 3.5}

Linux jest dynamicznie rozwijającym się projektem wolnego
oprogramowania z~dość stabilnym harmonogramem wydawania nowych wersji
--- średnio 2-3 miesiące\footnote{Konkretnie, z~pośród 35 wydań
  Linuksa od 2.6.13 aż do 3.7, jedynie trzy potrzebowały więcej niż 92
  dni, konkretnie wersja 2.6.18 (94 dni), 2.6.24 (107 dni) i~3.1 (94
  dni).}.  W~połączeniu z~faktem, że alokator CMA budzi coraz większe
zainteresowanie, powoduje to, iż gdy już znalazł się w~oficjalnym
wydaniu, więcej osób zaczęło publikować poprawki.

\textcite{patch:cma-discard} zaimplementował usprawnienie, które
w~istotny sposób skraca czas alokacji pamięci.  W~swoich testach
zauważył przyśpieszenie alokacji \unit[10]{MiB} z~\unit[146]{ms} do
zaledwie \unit[7]{ms}.  Pomysł polega na odrzucaniu stron które można
w~prosty sposób odzyskać.  Najprostszym przykładem są tutaj bufory
dyskowe --- ich zawartość można przywrócić ponownie odczytując dane
z~nośnika.

\section{Możliwe dalsze prace}

Nie spoczywając na laurach, obecnie rozważam kilka kolejnych
usprawnień.  Począwszy od stosunkowo prostych i~nieinwazyjnych, aż po
bardziej drastyczne zmiany w~kodzie, które częściowo zmieniają
zachowanie alokatora.

\subsection{Algorytm doboru stron}

Istotnym ulepszeniem może się okazać zmiana algorytmu doboru zakresu
stron (omówionego w~podrozdziale \ref{sec:alloc-from-contig}).  Metoda
„pierwszy pasujący” jest prosta w~zaimplementowaniu, ale chociażby
algorytm „najlepszy pasujący” mógłby zmniejszyć fragmentacje pamięci.

Co istotniejsze, funkcja \code|dma_alloc_from_contiguous| nie
uwzględnienia, które strony wymagają migracji.  Może to powodować, iż
alokator będzie dokonywał migracji, którym można było zapobiec.  Na
chwilę obecną nie wiadomo jednak, czy zysk z~nowego algorytmu nie
zostałby przysłonięty kosztami wynikającymi z~jego złożoności jak
i~możliwą większą fragmentacją.
n
\subsection{Strony, których nie można przenieść}

Innym dość uciążliwym problemem, z~którym CMA musi sobie radzić, jest,
fakt iż nie zawsze istnieje możliwość migracji pewnych stron
ruchomych.  Może to być spowodowane brakiem funkcji migrującej lub
chwilowym wykorzystaniem strony w~kontekście wymagającym stałego
adresu fizycznego.

Do pierwszej kategorii należą np.\ strony wykorzystywane przez wiele
systemów plików.  Nawet w~bardzo popularnym i~powszechnie używanym
systemie plików ext4, stron dziennika nie da się przenieść.
Naprawienie tego problemu jest niestety dość skomplikowane.  Wymaga to
bowiem wyszukania fragmentów, które powodują taki stan rzeczy,
a~znajdują się one w~wielu różnych podsystemach jądra i~aby dodać
brakujące funkcje migrujące należy choć w~podstawowym stopniu zapoznać
się z~kodem.

Druga kategoria to sytuacje, gdy strona została unieruchomiona na
czas, gdy wykonywany jest na niej transfer DMA (lub inna operacja
wymagająca stałego adresu fizycznego).  Przykładowo, w~sytuacji gdy
dane są kopiowane pomiędzy pamięcią a~dyskiem twardym.  Jednym
z~rozważanych przeze mnie rozwiązań było migrowanie strony poza region
CMA zanim zostanie ona unieruchomiona.  Niestety unieruchamianie stron
jest dość częstym zjawiskiem i~kopiowanie danych za każdym razem
prowadziłoby do znacznej degradacji wydajności systemu.


\subsection{Ograniczenie dozwolonych rodzajów stron}

Możliwą zmianą, która w~największym stopniu wpływa na mechanizm CMA
jest ograniczenie wykorzystywania regionów CMA tylko do stron
przeznaczonych dla funkcji, które zawsze można migrować.

Ciekawym przykładem jest tutaj zRam \autocite{patch:zram}, który jest
mechanizmem, pozwalającym na tworzenie partycji wymiany w~pamięci RAM.
Z~pozoru może się to wydawać dziwnym rozwiązaniem, ale istotnym
elementem jest kompresja danych.  Dzięki niej, po przeniesieniu stron
do takiej „partycji wymiany” zajmują one mniej miejsca w~pamięci.
Jednym z~zastosowań zRam są systemy wbudowane, które posiadają dyski
flash, które (z~uwagi na ograniczoną liczbę zapisów na medium) nie
nadają się do wykorzystania jako partycje wymiany.

Pamięć transcendentna \autocite{bib:transcendent} (działająca
w~przestrzeni jądra) i mechanizm \code|POSIX_FADV_VOLATILE|
\autocite{bib:fadv-volatile} (działający w~przestrzeni użytkownika)
również posiadają ciekawe właściwości.  W~obu przypadkach, ideą jest
oznaczenie pewnych danych jako ulotne, co oznacza, że w~sytuacji, gdy
brakuje wolnej pamięci w~systemie, jądro może takie dane zwyczajnie
porzucić.

Jednym z~flagowych zastosowań jest pamięć podręczna przeglądarek
internetowych.  Przechowywanie danych w~pamięci przeglądarki może
w~znacznym stopniu poprawić doświadczenie użytkownika, ale z~drugiej
strony większe zużycie pamięci może spowolnić resztę środowiska.
Dzięki zastosowaniu \code|POSIX_FADV_VOLATILE| jądro może zdecydować,
aby usunąć dane z~pamięci podręcznej przeglądarki.  Oczywiście z~puntu
widzenia CMA, możliwość usunięcia danych z~pamięci jest bardzo
kusząca.

W~przypadku takich rozwiązań problemem może być zbytnie ograniczenie
zastosowania stron z~regionów CMA.  Jeżeli niewiele rodzajów danych
może korzystać z~zarezerwowanej pamięci to może to prowadzić do
nieefektywnego jej wykorzystania, co sprowadza CMA z~powrotem do
punktu wyjścia.

\subsection{Podsumowanie}

Droga przed alokatorem CMA jest otwarta i~istnieje wiele aspektów nad
które można ulepszać, a~dzięki coraz większemu gronu osób
zainteresowanych tym kodem, można pokusić się o~predykcję, iż
mechanizm CMA będzie się rozwijał.  Równocześnie kolejne osoby
wykazuje zainteresowanie korzystania z~niego w~systemach znacznie
różniących się od platform, dla których był projektowany
(tj.\ telefonów komórkowych), takich jak zarządcy maszyn wirtualnych
jaki i~oprogramowanie samolotów.
