\chapter{Sposób użycia interfesju CMA}\label{sec:cma-usage}

Idea mechanizmu CMA jest stosunkowo prosta, jednak urósł on do raczej
skomplikowanego kawałka kodu, który integruje się dość głęboko
z~systemem zarządzania pamięci jądra Linuksa.  Pomimo tego, jego
użycie nie jest szczególnie trudne, a~w~wielu przypadkach autor
sterownika, który chciałby korzystać z~alokatora CMA nie musi niczego
zmieniać w~swoim kodzie.

\section{Wykorzystanie w~sterownikach}\label{sec:usage-drivers}

Alokator CMA integruje się z~interfejsem programowania DMA (DMA API),
co oznacza, że jeżeli mechanizm CMA jest włączony i~zintegrowany
z~daną architekturą, poprawnie napisany sterownik (tzn.\ taki, który
korzysta z~DMA API) będzie korzystał z~alokatora CMA bez konieczności
dokonywania jakichkolwiek zmian.

Tak naprawdę, sterowniki nie powinny odwoływać się bezpośrednio do
funkcji CMA, gdyż są one zbyt nisko poziomowe i~operują na stronach,
gdy tymczasem sterowniki urządzeń są raczej zainteresowane adresami
szyny.\footnote{W~ogólności, adres szyny strony może być inny niż jej
  adres fizyczny i~DMA API zostało zaprojektowane tak, aby brać to pod
  uwagę.}  Co więcej, mechanizm CMA nie posiada żadnych interfejsów
gwarantujących spójność pamięci podręcznej -- zadanie to leży
w~kwestii DMA API.

Najprostszym mechanizmem z~tego interfejsu są funkcje
\code|dma_alloc_coherent| i~\code|dma_free_coherent|.  Służą one
odpowiednio do alokacji i~zwalniania buforów DMA, których zawartość
jest zawsze spójna z~tym co widzi procesor\footnote{W~różnych
  architekturach efekt ten jest uzyskiwany w~różny sposób.
  W~architekturze Intel istnieje gwarancja spójność zawartości kości
  RAM oraz pamięci podręcznej procesora, gdy tymczasem architektura
  ARM nie daje takich gwarancji.  Z~tego powodu, w~systemach opartych
  o~Linuksa działających na platformie ARM, bufory alokowane przy
  pomocy \code|dma_alloc_coherent| nie podlegają buforowaniu w~pamięci
  podręcznej procesora.}.  Wydruk \ref{lst:dma-alloc-example} pokazuje
sposób wykorzystania tych dwóch funkcji.  \textcite{patch:cma-test}
stworzył prosty sterownik, który można wykorzystać do testowania
mechanizmu CMA.  Dokładniejszy opis DMA API można znaleźć w~rozdziale
15 \autocite{bib:ldd3}.

\begin{lstlisting}[float=tbhp,caption={Alokacja bufora DMA z~użyciem
      DMA API.},label=lst:dma-alloc-example]
static struct device *my_dev;

void *my_dev_alloc_buffer(unsigned long size_in_bytes, dma_addr_t *dma_addrp)
{
	void *virt_addr;

	virt_addr = dma_alloc_coherent(my_dev, size_in_bytes,
				       dma_addrp, GFP_KERNEL);
	if (!virt_addr)
		dev_err(my_dev, "Unable to allocate %lu-byte DMA %buffer",
			size_in_bytes);
	return virt_addr;
}

void *my_dev_free_buffer(unsigned long size_in_bytes,
			 void *virt_addr, dma_addr_t dma_addr)
{
	dma_free_coherent(my_dev, size_in_bytes, virt_addr, dma_addr);
}
\end{lstlisting}


\section{Integracja z~architekturą procesora}\label{sec:integrate-with-arch}

Alokator CMA działa dzięki rezerwowaniu w~trakcie startu systemu
pewnego regionu pamięci (zwanego regionem lub kontekstem CMA), który
po zainicjowaniu całego mechanizmu CMA jest zwracany do systemu (tak
że może być wykorzystywany do pewnego rodzaju alokacji).  Aby taki
obszar został zarezerwowany, w~trakcie startu systemu musi zostać
wywołana funkcja:

\begin{lstlisting}
void dma_contiguous_reserve(phys_addr_t limit);
\end{lstlisting}

Wywołanie to musi nastąpić gdy podsystem alokacji pamięci czasu startu
systemu (tj.\ \ang*{memblock}) zostanie zainicjowany, ale przed
aktywowaniem alokatora stron.  Dla przykładu w~architekturze ARM
dogodnym miejscem jest funkcja \code|arm_memblock_init|, a~x86 --
\code|setup_arch| zaraz po aktywowaniu memblock.

Argument \code|limit| określa górny adres pamięci fizycznej, którego
zarezerwowany obszar CMA nie przekroczy.  Dzięki niemu regiony CMA
mogą zostać ograniczone do adresów dostępnych dla urządzeń w~systemie.
Przykładowo w~architekturze ARM argument ten przyjmuje wartość
zmiennych \code|arm_dma_limit| lub \code|arm_lowmem_limit|,
którakolwiek jest mniejsza.  W~procesorach 64-bitowych może zaistnieć
potrzeba ograniczenia do 32-bitowych adresów.  Jeżeli wartością tego
argumentu jest zero, na kontekst CMA nie jest narzucany żaden limit.

Ilość zarezerwowanej pamięci zależy od argumentu \code|cma| (który
określa rozmiar regionu w~bajtach) przekazywanego do jądra w~trakcie
startu, lub, jeżeli argumentu tego nie ma, ustawień kompilacji jądra.
W~trakcie konfiguracji jądra można wybrać jeden z~czterech sposobów
określania rozmiaru:

\begin{enumerate}
\item stały rozmiar wyrażony w~bajtach,
\item rozmiar wyrażony w~procentach całkowitej pamięci dostępnej
  w~systemie,
\item większe z~pierwszych dwóch opcji, lub
\item mniejsze z~pierwszych dwóch opcji.
\end{enumerate}

Domyślną wartością konfiguracji jest alokacja \unit[16]{MiB}.

Funkcja \code|dma_contiguous_reserve| tworzy domyślny region CMA
wykorzystywany przez wszystkie urządzenia, które nie mają przypisanych
prywatnych kontekstów.  Prywatne regiony opisane są w~podrozdziale
\ref{sec:priv-regions}.


\subsection{Poprawki specyficzne dla architektury}

Na niektórych architekturach może zaistnieć przeprowadzenia dodatkowej
obróbki zarezerwowanych regionów pamięci.  Przykładowo, z~uwagi na
brak gwarancji spójności pamięci RAM i~pamięci podręcznej procesora na
architekturze ARM \autocite[podrozdział B5.5]{bib:arm-arch-reference},
wymagane jest aby strony, z~których korzystają urządzenia, były
mapowane jako niepodlegające buforowaniu (\ang{noncacheable}).  Co
więcej, specyfikacja architektury mówi, że jeżeli dana strona jest
mapowana z~różnymi parametrami buforowania (\ang{cacheability}), efekt
działania systemu nie jest zdefiniowany.

Dlatego w~architekturze ARM, kod integrujący CMA z~DMA API zmienia
mapowanie strony na niechachewalne na czas, gdy jest ona używana przez
urządzenie.  Z~drugiej strony, aby przyśpieszyć translację adresów,
jądro stara się stosować tak zwane wielkie strony.  Pozwala to
zmapować \unit[2]{MiB} pamięci (512 normalnych stron o~rozmiarze
\unit[4]{KiB}) poprzez jeden wpis w~tablicy mapowania.

Niestety, takie mapowanie uniemożliwia zmianę parametrów mapowania
pojedynczej strony.  Z~uwagi na to, regiony CMA są przygotowane w~ten
sposób, że mapowanie wielkich stron jest rozbijane na wiele mapowań
pojedynczych stron co pozwala na (w~miarę) proste modyfikowanie
ustawień cachowania danej strony.  Więcej na ten temat napisał
\textcite{bib:cma-and-arm}.

Aby to umożliwić, dla każdego kontekstu CMA, zawołana zostanie funkcja:

\begin{lstlisting}
void dma_contiguous_early_fixup(phys_addr_t base, unsigned long size);
\end{lstlisting}

Nie jest ona zaimplementowana przez alokator CMA i~musi zostać
dostarczona wraz z~kodem danej architektury.  Jej deklaracja powinna
znaleźć się w~pliku nagłówkowym \code|asm/dma-contiguous.h|.  Jeżeli
funkcjonalność ta nie jest konieczna, wystarczy dostarczyć pustą
implementację.

Należy pamiętać, że funkcja ta jest wołana dość wcześnie w~trakcie
startu systemu, zatem wiele podsystemów może jeszcze nie być
dostępnych, a~w~szczególności funkcja \code|kmalloc| nie będzie
działać.  Co więcej, może ona zostać wywołana kilkakrotnie, dla
różnych regionów CMA, ale nie więcej niż \code|MAX_CMA_AREAS| razy
(domyślnie osiem).

\subsection{Integracja z~podsystemem DMA}\label{sec:usage-integrate}

Aby sterowniki mogły korzystać z~alokatora CMA poprzez DMA API, CMA
musi zostać dodane do podsystemu DMA danej architektury.  Alokacja
buforu CMA odbywa się poprzez wywołanie funkcji:

\begin{lstlisting}
struct page *dma_alloc_from_contiguous(
	struct device *dev,
	int count,
	unsigned int align);
\end{lstlisting}

Pierwszym argumentem jest urządzenie na rzecz którego odbywa się
alokacja.  Drugim jest \emph{liczba stron} do zaalokowania.

Trzeci argument to wyrównanie alokacji wyrażone jako~rzęd strony.
Innymi słowy jeżeli bufor ma być wyrównany do $a$ bajtów, parametr
\code|align| powinien przyjąć wartość $\log_2 a - \log_2
\mathrm{PAGE\_SIZE}$ (co dla stron o~rozmiarze \unit[4096]{KiB}
oznacza $\log_2 a - 12$).  Jeżeli żadne wyrównanie nie jest wymagane,
należy zwyczajnie przekazać zero -- zmniejszy to również problem
z~fragmentacją.  Warto zauważyć, że na wartość argumentu \code|align|
nałożone jest z~góry ograniczenie \code|CONFIG_CMA_ALIGNMENT|.  Jego
domyślną wartością jest osiem (co oznacza wyrównanie do 256 stron).

Funkcja \code|dma_alloc_from_contiguous| zwraca wskaźnik na pierwszą
stronę spośród serii \code|count| zaalokowanych stron lub \code|NULL|
w~przypadku nieudanej alokacji.

Do zwolnienia bufora wykorzystywana jest funkcja:

\begin{lstlisting}
bool dma_release_from_contiguous(
	struct device *dev,
	struct page *pages,
	int count);
\end{lstlisting}

Argumenty \code|dev| i \code|count| mają takie samo
znaczenie jak w~funkcji \code|dma_alloc_from_contiguous|,
a~argument \code|pages| jest wartością zwróconą przez tę funkcję.

Jeżeli dany bufor nie był zaalokowany poprzez interfejs CMA, funkcja
zwróci \code|false|, w~przeciwnym wypadku, bufor zostanie zwolniony
i~funkcja zwróci \code|true|.  Zwracana wartość może zostać
wykorzystana, aby rozróżnić, czy dany bufor był buforem CMA, czy też
nie.

Wydruk \ref{lst:dma-integration} pokazuje fragment kodu, który
integruje mechanizm CMA z~podsystemem DMA architektury x86.  Warto
zwrócić uwagę, jak w~funkcji \code|dma_generic_free_coherent| wartość
zwracana przez \code|dma_release_from_contiguous| jest wykorzystana,
aby podjąć decyzję, czy należy zwolnić bufor korzystając z~funkcji
\code|free_pages|.

\begin{lstlisting}[float=bht,caption={Integracja alokatora CMA z~podsystemem DMA
      architektury x86.},label=lst:dma-integration]
diff --git a/arch/x86/kernel/pci-dma.c b/arch/x86/kernel/pci-dma.c
@@ -99,14 +99,18 @@ void *dma_generic_alloc_coherent(
 				 dma_addr_t *dma_addr, gfp_t flag)
 {
	(*{\it [ \ldots ]}*)
 again:
-	page = alloc_pages_node(dev_to_node(dev), flag, get_order(size));
+	if (!(flag & GFP_ATOMIC))
+		page = dma_alloc_from_contiguous(dev, count, get_order(size));
+	if (!page)
+		page = alloc_pages_node(dev_to_node(dev), flag, get_order(size));
 	if (!page)
 		return NULL;
@@ -126,6 +130,16 @@ again:
 	return page_address(page);
 }

+void dma_generic_free_coherent(struct device *dev, size_t size, void *vaddr,
+			       dma_addr_t dma_addr)
+{
+	unsigned int count = PAGE_ALIGN(size) >> PAGE_SHIFT;
+	struct page *page = virt_to_page(vaddr);
+
+	if (!dma_release_from_contiguous(dev, page, count))
+		free_pages((unsigned long)vaddr, get_order(size));
+}
+
\end{lstlisting}

Funkcja \code|dma_alloc_from_contiguous| nie może zostać wywołana
w~kontekście atomowym (np.\ z~procedury obsługi przerwania),
a~jednocześnie dopuszczalne jest wywołanie \code|dma_alloc_coherent|
z~takiego kontekstu.  Z~tego powodu, podsystem DMA musi posiadać inny
mechanizm przeznaczony dla takich alokacji.  Najprostszym rozwiązaniem
jest zarezerwowanie pewnego, stosunkowo niewielkiego, obszaru pamięci,
przeznaczonego do alokacji w~kontekście atomowym.  Istniejące
architektury muszą posiadać tego typu mechanizmy.


\section{Regiony CMA dla poszczególnych urządzeń}\label{sec:priv-regions}

Po dokonaniu zmian opisanych w~powyższym podrozdziale, sterowniki
urządzeń powinny już działać.  Korzystając z~interfesju DMA odwołują
się bowiem do alokatora CMA.

Jednak niektóre urządzenia mogą mieć specyficzne wymagania.
Wspomniany już w~podrozdziale \ref{sec:evo-cma} koder multimedialny
wymaga, aby bufory na różne dane, znajdowały się w~różnych bankach
pamięci.  Ponadto, zależnie od istniejących na platformie urządzeń,
wskazane może być izolowanie pewnych grup urządzeń.  Dla przykładu
mieszanie alokacji dla stosunkowo małych tekstur dla koprocesora
graficznego z~alokacjami dużych buforów przeznaczonych dla kamery,
może przyczynić się do zwiększenia fragmentacji.

Funkcja \code|dma_declare_contiguous| tworzy domyślny kontekst CMA,
ale istnieje możliwość przypisania różnych regionów do różnych
urządzeń.  Istnieje mapowanie wiele-do-jednego pomiędzy strukturą
\code|device|, a~kontekstem CMA.  Oznacza to, że pojedynczy region CMA
może zostać przypisany do danego urządzenia, ale jeżeli urządzenie ma
korzystać z~wielu kontekstów CMA konieczne jest stworzenie kilku
struktur \code|device|.

Aby przypisać region CMA do urządzenia wystarczy wywołać funkcję:

\begin{lstlisting}
int dma_declare_contiguous(
	struct device *dev,
	unsigned long size,
	phys_addr_t base,
	phys_addr_t limit);
\end{lstlisting}

Pierwszy argument to urządzenie do którego kontekst ma być przypisany.
Drugi to \emph{rozmiar w~bajtach}.  Trzeci to adres gdzie region ma
się zaczynać lub zero, jeżeli nie ma to znaczenia.  Ostatni argument,
\code|limit|, ma takie samo znaczenie jak w~przypadku funkcji
\code|dma_contiguous_reserve|.  Dla przykładu, wydruk
\ref{lst:s5p-priv-region} pokazuje fragment kodu dodającego prywatne
konteksty do dwóch urządzeń.

Istnieje limit liczby „prywatnych” regionów CMA.  Konkretnie może być
ich co najwyżej \code|CONFIG_CMA_AREAS| (domyślnie siedem).  Jeżeli
limit ten zostanie przekroczony, funkcja \code|dma_declare_contiguous|
zacznie zwracać \code|-ENOSPC|.  Jeżeli istnieje taka potrzeba, nic
nie stoi na przeszkodzie aby ten limit zwiększyć w~trakcie kompilacji
jądra.

% For some reason, breaklines=true does not work, so I'm breaking the
% line manually...
\begin{lstlisting}[float=tbhp,caption={Przypisanie prywatnych regionów
      CMA do dwóch urządzeń.},label=lst:s5p-priv-region]
diff --git a/arch/arm/plat-s5p/dev-mfc.c b/arch/arm/plat-s5p/dev-mfc.c

 void __init s5p_mfc_reserve_mem(phys_addr_t rbase, unsigned int rsize,
 				phys_addr_t lbase, unsigned int lsize)
 {
	(*{\it [ \ldots ]}*)
+	if (dma_declare_contiguous(&s5p_device_mfc_r.dev, (*{\color{gray} $\hookleftarrow$}*)
(*{\color{gray} $\hookrightarrow$}*)		rsize, rbase, 0))
+		printk(KERN_ERR "Failed to reserve memory for MFC device (*{\color{gray} $\hookleftarrow$}*)
(*{\color{gray} $\hookrightarrow$}*)			(%u bytes at 0x%08lx)\n",
+		       rsize, (unsigned long) rbase);
	(*{\it [ \ldots ]}*)
+	if (dma_declare_contiguous(&s5p_device_mfc_l.dev, (*{\color{gray} $\hookleftarrow$}*)
(*{\color{gray} $\hookrightarrow$}*)		lsize, lbase, 0))
+		printk(KERN_ERR "Failed to reserve memory for MFC device (*{\color{gray} $\hookleftarrow$}*)
(*{\color{gray} $\hookrightarrow$}*)			(%u bytes at 0x%08lx)\n",
+		       rsize, (unsigned long) rbase);
 }
\end{lstlisting}

Odrobinę bardziej skomplikowane jest przypisanie tego samego kontekstu
do kilku urządzeń.  Obecny interfejs CMA nie udostępnia funkcji, która
by na to pozwalała, ale i~tak nie jest to szczególnie trudne do
osiągnięcia.  Wystarczy zastosować metodę opisaną powyżej, aby
przypisać region do jednego urządzenia, a~następnie skopiować ten
region do drugiego urządzenia.  Całą sekwencja powinna zostać wykonana
jako \code|postcore_initcall|.  Poniższy kod pokazuje jak taki efekt
może zostać osiągnięty:

\begin{lstlisting}
static int __init foo_set_up_cma_areas(void) {
	struct cma *cma = dev_get_cma_area(device1);
	dev_set_cma_area(device2, cma);
	return 0;
}
postcore_initcall(foo_set_up_cma_areas);
\end{lstlisting}

Warto zauważyć, że nic nie stoi na przeszkodzie, aby nie tworzyć
domyślnego kontekstu CMA.  Oczywiście, jeżeli nie zostanie on
stworzony, urządzenia, którym nie zostaną przypisane prywatne regiony
nie będą mogły korzystać z~buforów CMA.
