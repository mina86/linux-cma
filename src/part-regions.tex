\Section{CMA regions}{Private \& not so private CMA regions}

\subsection{Default CMA regions}

\begin{frame}
  \frametitle{Default CMA region}

  \begin{itemize}
  \item Memory reserved for CMA is called CMA region or CMA context.
  \item There's one default context devices use.
  \item So why \lstinline|dma_alloc_from_contiguous()| take device as
    argument?
  \item There may also be per-device, private context.
  \end{itemize}
\end{frame}

\subsection{Private CMA regions}

\begin{frame}
  \frametitle{What is private region for?}

  \begin{itemize}
  \item Separate a~device into its own pool.
    \begin{itemize}
    \item May help with fragmentation.
    \item For instance big vs small allocations.
    \item Several devices may be grouped together.
    \end{itemize}
  \item Use different contexts for different purposes within the same
    device.
    \begin{itemize}
    \item Simulating dual channel memory.
    \item Big and small allocations in the same device.
    \end{itemize}
  \end{itemize}
\end{frame}

\begin{frame}[fragile]
  \frametitle{Declaring private regions}

  \begin{block}{Declaring private regions}
\begin{lstlisting}
int dma_declare_contiguous(
    struct device *dev,
    unsigned long size,
    phys_addr_t base,
    phys_addr_t limit);
\end{lstlisting}
  \end{block}

  \begin{description}[count]
  \item[dev] Device that will use this region.
  \item[size] \emph{Size in bytes pages} to allocate. {\footnotesize Not
    pagas nor order.}
  \item[base] Base address of the region (or zero to use anywhere).
  \item[limit] Upper limit of the region (or zero for no limit).
  \item Returns zero on success, negative error code on failure.
  \end{description}
\end{frame}

\begin{frame}[fragile]
  \frametitle{Declaring private regions}

  \begin{block}{Declaring private regions}
\begin{lstlisting}
int dma_declare_contiguous(
    struct device *dev,
    unsigned long size,
    phys_addr_t base,
    phys_addr_t limit);
\end{lstlisting}
  \end{block}

  \begin{description}[count]
  \item[dev] Device that will use this region.
  \item[size] \emph{Size in bytes pages} to allocate. {\footnotesize Not
    pagas nor order.}
  \item[base] Base address of the region (or zero to use anywhere).
  \item[limit] Upper limit of the region (or zero for no limit).
  \item Returns zero on success, negative error code on failure.
  \end{description}
\end{frame}

\subsection{Region sharing \& multiple regions per device}

\begin{frame}[fragile]
  \frametitle{Region shared by several devices}

  \begin{itemize}
  \item The API allows to assign a~region to a~single device.
  \item What if more than one device is to use the same region.
  \item It can be easily done via “copying” the context pointer.
  \end{itemize}

  \begin{block}{Copying CMA context pointer between two devices}
\begin{lstlisting}
static int __init foo_set_up_cma_areas(void) {
    struct cma *cma;
    cma = dev_get_cma_area(device1);
    dev_set_cma_area(device2, cma);
    return 0;
}
postcore_initcall(foo_set_up_cma_areas);
\end{lstlisting}
  \end{block}
\end{frame}

\begin{frame}
  \frametitle{Several regions used by the same device}

  \begin{itemize}
  \item CMA uses one-to-many mapping from \lstinline|device| structure
    to CMA region.
  \item As such, one device can only use one CMA context\ldots
  \item \ldots unless it uses more than one \lstinline|device|
    structure.
  \item That's exactly what S5PV110's MFC does.
  \end{itemize}
\end{frame}
