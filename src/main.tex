\documentclass{beamer}
\usepackage[utf8]{inputenc}
\usepackage[T1]{fontenc} % I've gotten "no %%Page comments generated" with this
\usepackage[english]{babel}
\usepackage{pgf}
\usepackage{units}
\usepackage{lmodern}
\usepackage{color}
\usepackage{listings}

\definecolor{gray}{rgb}{0.5,0.5,0.5}
\definecolor{darkgreen}{rgb}{0.0,0.5,0.0}
\lstset{
  language=C,
  basicstyle=\small,
  numbers=left,
  numberstyle=\tiny\color{gray},
  stepnumber=1,
  numbersep=5pt,
  showspaces=false,
  showstringspaces=false,
  showtabs=false,
  tabsize=8,
  captionpos=b,
  breaklines=true,
  breakautoindent,
  columns=fullflexible,
  morekeywords={inline},
  keywordstyle=\color{blue}\bfseries,
  commentstyle=\color{darkgreen}\itshape,
  stringstyle=\color{magenta}\ttfamily
}
\lstdefinelanguage{diff}{
  morecomment=[f][\color{blue}]{@@},     % group identifier
  morecomment=[f][\color{red}]-,         % deleted lines
  morecomment=[f][\color{darkgreen}]+,   % added lines
  morecomment=[f][\color{magenta}]{---}, % Diff header lines (must appear after +,-)
  morecomment=[f][\color{magenta}]{+++},
}

\usetheme{Frankfurt}

% Provide \CC to typeset "C++" better.
% http://www.parashift.com/c++-faq-lite/misc-environmental-issues.html#faq-40.2
\newcommand{\CC}{C\nolinebreak\hspace{-.05em}\raisebox{.4ex}{\footnotesize\bf +}\nolinebreak\hspace{-.10em}\raisebox{.4ex}{\footnotesize\bf +}}

% Define the Google colours.
\definecolor{GoogleBlue}{RGB}{0,102,204}
\definecolor{GoogleYellow}{RGB}{255,204,0}
\definecolor{GoogleGreen}{RGB}{0,153,0}
\definecolor{GoogleRed}{RGB}{255,0,0}
\definecolor{GoogleGray}{RGB}{89,89,89}
\newcommand\Google{\href{http://www.google.com/}{%
{\color{GoogleBlue}G}%
{\color{GoogleRed}o}%
{\color{GoogleYellow}o}%
{\color{GoogleBlue}g}%
{\color{GoogleGreen}l}%
{\color{GoogleRed}e}}}

% Use GoogleBlue for all structural text (headers, bullets, etc.)
\setbeamercolor{structure}{fg=GoogleBlue,bg=white}

% Shrink the margins slightly.
\setbeamersize{text margin left=5mm}
\setbeamersize{text margin right=5mm}

% Remove navigation symbols. They are never used anyway.
\setbeamertemplate{navigation symbols}{}

% Make more room on notes pages (when used).
\setbeamertemplate{note page}[plain]
\setbeamerfont{note page}{size=\scriptsize}

% Gray footnotes, no rule.
% FIXME: This is broken. The gray continues onto subsequent slides.
% Note: footnotes are generally a bad idea in presentations.
\setbeamercolor{footnote mark}{fg=GoogleGray}
\renewcommand{\footnoterule}{}

\newcommand\gheaderheight{1mm}
\setbeamertemplate{footline}{%
    \begin{pgfpicture}{0cm}{0cm}{\textwidth}{\gheaderheight}
    \pgfsetcolor{GoogleBlue}
    \pgfrect[fill]{\pgfpoint{0.0\textwidth}{0}}
                  {\pgfpoint{0.25\textwidth}
                  {\gheaderheight}}
    \pgfsetcolor{GoogleRed}
    \pgfrect[fill]{\pgfpoint{0.25\textwidth}{0}}
                  {\pgfpoint{0.25\textwidth}
                  {\gheaderheight}}
    \pgfsetcolor{GoogleYellow}
    \pgfrect[fill]{\pgfpoint{0.50\textwidth}{0}}
                  {\pgfpoint{0.25\textwidth}
                  {\gheaderheight}}
    \pgfsetcolor{GoogleGreen}
    \pgfrect[fill]{\pgfpoint{0.75\textwidth}{0}}
                  {\pgfpoint{0.25\textwidth}
                  {\gheaderheight}}
    \end{pgfpicture}
}


\newcommand\Section[2]{%
\section[#1]{#2}%
\begin{frame}%
\tableofcontents[currentsection,sections=\thesection]%
\end{frame}}


\title{Continuous Memory Allocator}
\author[Michał Nazarewicz]{%
  \texorpdfstring{Michał Nazarewicz\vskip 8pt%
    \scriptsize\href{mailto:mina86@mina86.com}{mina86@mina86.com}}{%
    Michał Nazarewicz}}
\institute{\Google}
\date{\today}

\setbeamerfont{institute}{size={\fontsize{10pt}{12pt}}}
\setbeamerfont{date}{size={\fontsize{8pt}{10pt}}}


\begin{document}

\begin{frame}
  \titlepage
\end{frame}

\begin{frame}
  \frametitle{Who am I?}

  \begin{itemize}
  \item Just a~random FOSS enthusiast.
  \item 2009--2010 at Samsung.
    \begin{itemize}
    \item USB gadgets.
    \item Memory management.
    \end{itemize}
  \item 2011--now at Google.
  \end{itemize}
\end{frame}

\begin{frame}
  \frametitle{Table of Contents}
  \tableofcontents[hideallsubsections]
\end{frame}

\Section{Introduction}{Why physically contiguous memory is needed}

\subsection{The mighty MMU}
\begin{frame}
  \frametitle{The mighty MMU}

  \begin{itemize}
  \item Modern CPUs have MMU.
    \begin{itemize}
    \item Virtual $\rightarrow$ physical address.
    \end{itemize}
  \item Virtually contiguous $\notimplies$ physically contiguous.
  \item So why bother?
  \end{itemize}
\end{frame}

\subsection{System devices}
\begin{frame}
  \frametitle{System devices}

  \begin{itemize}
  \item MMU stands behind CPU.
  \item There are other chips in the system.
  \item Some require large buffers.
    \begin{itemize}
    \item 5-megapixel camera anyone?
    \end{itemize}
  \item On embedded, there's plenty of those.
  \end{itemize}
\end{frame}

\subsection{The mighty DMA}
\begin{frame}
  \frametitle{The mighty DMA}

  \begin{itemize}
  \item DMA can do vectored I/O.
  \item Gathering buffer from scattered parts.
  \item Contiguous for the device $\notimplies$ physically contiguous.
  \item So why bother?
  \end{itemize}
\end{frame}

\subsection{The mighty system MMU}
\begin{frame}
  \frametitle{The mighty system MMU}

  \begin{itemize}
  \item What about a~system MMU?
    \begin{itemize}
    \item Address coming from the device $\rightarrow$ physical
      address
    \end{itemize}
  \item Same deal as with CPU's MMU.
  \item So why bother?
  \end{itemize}
\end{frame}

\subsection{Cost, speed \& power}
\begin{frame}
  \frametitle{Cost, speed \& power}

  \begin{itemize}
  \item Every chip costs.
    \begin{itemize}
    \item Money and power.
    \end{itemize}
  \item More complex chips cost more.
  \item Not all systems have DMA with SG or system MMU.
  \item System MMU takes time.
  \end{itemize}
\end{frame}

\section[Solutions]{Solutions prior to CMA}

\subsection{Lie to the kernel}
\begin{frame}
  \frametitle{Lie to the kernel}

  \begin{itemize}
  \item Lie to the kernel about amount of memory.
    \begin{itemize}
    \item Easily done with a~mem parameter.
    \item Kernel won't touch memory hidden via mem.
    \end{itemize}
  \item Assign buffers to each device that might need it.
  \item Platform dependent.
  \item Requires fiddling with boot loader.
  \item \ldots
  \end{itemize}
\end{frame}

\subsection{Reserve and assign at boot time}
\begin{frame}
  \frametitle{Reserve and assign at boot time}

  \begin{itemize}
  \item Reserve memory during system boot time.
  \item Assign buffers to each device that might need it.
  \item Less platform dependent.
  \item Boot loader is left alone.
  \item While device is not being used, memory is wasted.
  \end{itemize}
\end{frame}

\subsection{Reserve but allocate on demand}
\begin{frame}
  \frametitle{Reserve but allocate on demand}

  \begin{itemize}
  \item Reserve memory during system boot time.
  \item Provide API for allocating from that reserved pool.
  \item Less memory is reserved.
  \item But it's still wasted.
  \end{itemize}

  \begin{itemize}
  \item bigphysarea
  \item Physical Memory Manager
  \end{itemize}
\end{frame}

\subsection{Reserve but give back}
\begin{frame}
  \frametitle{Reserve but give back}

  \begin{itemize}
  \item Reserve memory during system boot time.
  \item Give it back
    \begin{itemize}
    \item but set it up so only movable pages can be allocated.
    \end{itemize}
  \item Provide API for allocating from that reserved pool.
  \item Migrate pages on allocation.
  \end{itemize}

  \begin{itemize}
  \item Contiguous Memory Allocator
  \end{itemize}
\end{frame}

\Section{Usage}{Continuous Memory Allocator Usage}

\subsection{Device drivers}

\begin{frame}
  \frametitle{Device drivers}

  \begin{itemize}
  \item CMA is integrated with the DMA API.
  \item If device driver uses the DMA API, nothing needs to be changed.
  \item In fact, device driver should always use the DMA API and never
    call CMA directly.
  \end{itemize}
\end{frame}

\begin{frame}[fragile]
  \frametitle{Device drivers}

  \begin{block}{Allocation}
\begin{lstlisting}
void *my_dev_alloc_buffer(
    unsigned long size_in_bytes, dma_addr_t *dma_addrp)
{
    void *virt_addr = dma_alloc_coherent(
        my_dev, size_in_bytes, dma_addrp, GFP_KERNEL);
    if (!virt_addr)
        dev_err(my_dev, "Allocation failed.");
    return virt_addr;
}
\end{lstlisting}
  \end{block}

\end{frame}

\begin{frame}[fragile]
  \frametitle{Device drivers}

  \begin{block}{Freeing}
\begin{lstlisting}
void *my_dev_free_buffer(
    unsigned long size, void *virt, dma_addr_t dma)
{
    dma_free_coherent(my_dev, size, virt, dma);
}
\end{lstlisting}
  \end{block}
\end{frame}

\subsection{Integration with the architecture}

\begin{frame}
  \frametitle{Integration with the architecture}

  \begin{itemize}
  \item CMA needs to be integrated with the architecture.
  \item Memory reservation.
  \item Early fixups.
  \item Integration with the DMA API.
  \item Let it compile!
  \end{itemize}
\end{frame}

\subsection{Memory reservation}

\begin{frame}
  \frametitle{Memory reservation}

  \begin{itemize}
  \item \lstinline|memblock| must be ready, page allocator must not.
  \item On ARM, \lstinline|arm_memblock_init()| is a~good place.
  \item All one needs to do, is to call
    \lstinline|dma_contiguous_reserve()|.
  \end{itemize}
\end{frame}

\begin{frame}[fragile]
  \frametitle{Memory reservation}

  \begin{block}{Reserving memory on ARM}
\begin{lstlisting}[language=diff]
 if (mdesc->reserve)
     mdesc->reserve();

+/*
+ * reserve memory for DMA contigouos allocations,
+ * must come from DMA area inside low memory
+ */
+dma_contiguous_reserve(min(arm_dma_limit, arm_lowmem_limit));
+
 arm_memblock_steal_permitted = false;
 memblock_allow_resize();
 memblock_dump_all();
\end{lstlisting}
  \end{block}
\end{frame}

\subsection{Early fixups}

\begin{frame}
  \frametitle{Early fixups}

  \begin{itemize}
  \item Kernel linear mapping uses huge pages.
  \item On ARM cache is not coherent.
  \item Having two mappings with different cache-ability gives
    undefined behaviour.
  \item So on ARM an “early fixup” is needed.
    \begin{itemize}
    \item This fixup alters the linear mapping so CMA regions use
      \unit[4]{KiB} pages.
    \end{itemize}
  \item The fixup is defined in
    \lstinline|dma_contiguous_early_fixup()| function
    \begin{itemize}
    \item which architecture needs to provide
    \item with declaration in a~\lstinline|asm/dma-contiguous.h| header file.
    \end{itemize}
  \end{itemize}
\end{frame}

\begin{frame}[fragile]
  \frametitle{Early fixups}

  \begin{block}{No need for early fixups}
\begin{lstlisting}
#ifndef ASM_DMA_CONTIGUOUS_H
#define ASM_DMA_CONTIGUOUS_H
#ifdef __KERNEL__

#include <linux/types.h>
#include <asm-generic/dma-contiguous.h>

static inline void
dma_contiguous_early_fixup(phys_addr_t base, unsigned long size)
{
    /* nop, no need for early fixups */
}

#endif
#endif
\end{lstlisting}
  \end{block}
\end{frame}

\subsection{Integration with the DMA API}

\begin{frame}
  \frametitle{Integration with DMA API}

  \begin{itemize}
  \item The DMA API needs to be modified to use CMA.
  \item CMA most likely won't be the only one.
  \end{itemize}
\end{frame}

\begin{frame}[fragile]
  \frametitle{Integration with DMA API}

  \begin{block}{Allocate}
\begin{lstlisting}
struct page *dma_alloc_from_contiguous(
    struct device *dev,
    int count,
    unsigned int align);
\end{lstlisting}
  \end{block}

  \begin{description}[count]
  \item[dev] Device the allocation is performed on behalf of.
  \item[count] \emph{Number of pages} to allocate. {\footnotesize Not
    number of bytes nor order.}
  \item[align] Order which to align to.  Limited by Kconfig option.
  \item Returns page that is the first page of \lstinline|count|
    allocated pages. {\footnotesize It's not a~compound page.}
  \end{description}
\end{frame}

\begin{frame}[fragile]
  \frametitle{Integration with DMA API}

  \begin{block}{Release}
\begin{lstlisting}
bool dma_release_from_contiguous(
    struct device *dev,
    struct page *pages,
    int count);
\end{lstlisting}
  \end{block}

  \begin{description}[count]
  \item[dev] Device the allocation was performed on behalf of.
  \item[pages] The first of allocated pages. {\footnotesize As
    returned on allocation.}
  \item[count] Number of allocated pages to allocate.
  \item Returns \lstinline|true| if memory was freed (ie.\ was managed
    by CMA) or \lstinline|false| otherwise.
  \end{description}
\end{frame}

\subsection{Let it compile!}

\begin{frame}
  \frametitle{Let it compile!}

  \begin{itemize}
  \item There's one think that needs to be done in \lstinline|Kconfig|.
  \item Architecture needs to \lstinline|select HAVE_DMA_CONTIGUEUS|.
  \item Without it, CMA won't show up under “Generic Driver Options”.
  \item Architecture may also \lstinline|select CMA| to force CMA in.
  \end{itemize}
\end{frame}

\subsection{Private \& not so private CMA regions}

\begin{frame}
  \frametitle{Default CMA region}

  \begin{itemize}
  \item Memory reserved for CMA is called CMA region or CMA context.
  \item There's one default context devices use.
  \item So why does \code{dma_alloc_from_contiguous()} take
    device as an argument?
  \item There may also be per-device or private contexts.
  \end{itemize}
\end{frame}

\begin{frame}
  \frametitle{What is a~private region for?}

  \begin{itemize}
  \item Separate a~device into its own pool.
    \begin{itemize}
    \item May help with fragmentation.
    \item For instance big vs small allocations.
    \item Several devices may be grouped together.
    \end{itemize}
  \item Use different contexts for different purposes within the same
    device.
    \begin{itemize}
    \item Simulating dual channel memory.
    \item Big and small allocations in the same device.
    \end{itemize}
  \end{itemize}
\end{frame}

\begin{frame}[fragile]
  \frametitle{Declaring private regions}

  \begin{block}{Declaring private regions}
\begin{lstlisting}
int dma_declare_contiguous(
    struct device *dev,
    unsigned long size,
    phys_addr_t base,
    phys_addr_t limit);
\end{lstlisting}
  \end{block}

  \begin{description}[countAA]
  \item[{\ttfamily dev}] Device that will use this region.
  \item[{\ttfamily size}] \emph{Size in bytes} to
    allocate. {\footnotesize Not pagas nor order.}
  \item[{\ttfamily base}] Base address of the region (or zero to use
    anywhere).
  \item[{\ttfamily limit}] Upper limit of the region (or zero for no
    limit).
  \item Returns zero on success, negative error code on failure.
  \end{description}
\end{frame}

\begin{frame}[fragile]
  \frametitle{Region shared by several devices}

  \begin{itemize}
  \item The API allows to assign a~region to a~single device.
  \item What if more than one device is to use the same region.
  \item It can be easily done via “copying” the context pointer.
  \end{itemize}
\end{frame}

\begin{frame}[fragile]
  \frametitle{Region shared by several devices, cont}

  \begin{block}{Copying CMA context pointer between two devices}
\begin{lstlisting}
static int __init foo_set_up_cma_areas(void) {
    struct cma *cma;
    cma = dev_get_cma_area(device1);
    dev_set_cma_area(device2, cma);
    return 0;
}
postcore_initcall(foo_set_up_cma_areas);
\end{lstlisting}
  \end{block}
\end{frame}

\begin{frame}
  \frametitle{Several regions used by the same device}

  \begin{itemize}
  \item CMA uses a~one-to-many mapping from \code{device} structure
    to CMA region.
  \item As such, one device can only use one CMA context\ldots
  \item \ldots unless it uses more than one \code{device}
    structure.
  \item That's exactly what S5PV110's MFC does.
  \end{itemize}
\end{frame}


\appendix

\section*{Q \& A}
\begin{frame}
  \frametitle{Q \& A}

  \begin{center}
  {\Huge ?}
  \end{center}
\end{frame}

\end{document}
