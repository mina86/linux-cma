\subsection{How does CMA work?}

\begin{frame}[fragile]
  \frametitle{Pages and page blocks}

  \begin{itemize}
  \item Linux manages memory in units of pages.
    \begin{itemize}
    \item Typically \unit[4]{KiB} in size.
    \end{itemize}
  \item Page can have order ranging from 0 to 10.\footnote{Strictly
    speaking, from zero to one less than \lstinline|MAX_ORDER| which is
    usually 11.}
    \begin{itemize}
    \item $n$-order page consists of $2^n$ \unit[4]{KiB} pages.
    \item 10-order page is called max-order page.
    \end{itemize}
  \item Pages are grouped into page blocks.
  \item Page block consists of 1024 pages, same size as max-order
    page.\footnote{This actually depends, but it's the case for ARM
      and x86.}
  \end{itemize}
\end{frame}

\begin{frame}[fragile]
  \frametitle{Migrate types}

  \begin{itemize}
  \item Each free page and each page block has a~migration type
    assigned to it.
  \item It does not change any properties of the page.
  \item Users specify migrate type on allocations.
  \item Page allocator tries to fulfil the requirements from pages of
    give type.
  \item If there are no free pages with given type, fallback types are
    used.
  \item For our purpose, let's say there are two possible types:
    \begin{description}[unmovable]
    \item[unmovable] Page cannot be migrated.
    \item[movable]   Page can be migrated.
    \end{description}
  \end{itemize}
\end{frame}

\begin{frame}
  \frametitle{Buddy allocator}

  \begin{itemize}
  \item Page allocator uses buddy allocation algorithm.
    \begin{itemize}
    \item Hence different names: buddy system or buddy allocator.
    \end{itemize}
  \item
  \end{itemize}
\end{frame}
