% Copyright (c) 2012 by Michał Nazarewicz <mina86@mina86.com>
% Distributed under the terms of the Creative Commons
% Attribution-ShareAlike 3.0 Unported (CC BY-SA 3.0) license.

\subsection{CMA implementation}

\begin{frame}[fragile]
  \frametitle{Interaction of CMA with Linux allocators}

  \begin{center}
    \includegraphics[width=\textwidth]{build/linux-allocators--cma.eps}
  \end{center}
\end{frame}

\begin{frame}
  \frametitle{CMA migrate type}

  \begin{itemize}
  \item CMA needs guarantees that large number of contiguous pages can
    be migrated.
    \begin{itemize}
    \item 100\% guarantee is of course never possible.
    \end{itemize}
  \end{itemize}

  \begin{itemize}
  \item CMA introduced a~new migrate type.
    \begin{itemize}
    \item \code{MIGRATE_CMA}
    \end{itemize}
  \item This migrate type has the following properties:
    \begin{itemize}
    \item CMA pageblocks never change migrate type.\footnote{Other
      than while CMA is allocating memory from them.}
    \item Only movable pages can be allocated from CMA pageblocks.
    \end{itemize}
  \end{itemize}
\end{frame}

\begin{frame}
  \frametitle{Preparing CMA region}

  \begin{itemize}
  \item At the boot time, some of the memory is reserved.
  \item When page allocator initialises, that memory is released with
    CMA's migrate type.
  \item This way, it can be used for movable pages.
    \begin{itemize}
    \item Unless the memory is allocated to a~device driver.
    \end{itemize}
  \item Each CMA region has a~bitmap of “CMA free” pages.
    \begin{itemize}
    \item “CMA free” page is one that is not allocated for device
      driver.
    \item It may still be allocated as a~movable page.
    \end{itemize}
  \end{itemize}
\end{frame}

\begin{frame}
  \frametitle{Allocation}

  \begin{center}
    \includegraphics[width=\textwidth]{build/cma-alloc-algo.eps}
  \end{center}

\end{frame}

\begin{frame}
  \frametitle{Migration}

  \begin{itemize}
  \item Pages allocated as movable are set up so that they can be
    migrated.
    \begin{itemize}
    \item Such pages are only references indirectly.
    \item Examples are anonymous process pages and disk cache.
    \end{itemize}
  \item Roughly speaking, migration consists of:
    \begin{enumerate}
    \item allocating a~new page,
    \item copying contents of the old page to the new page,
    \item updating all places where old page was referred, and
    \item freeing the old page.
    \end{enumerate}
  \item In some cases, content of movable page can simply be
    discarded.
  \end{itemize}

\end{frame}
